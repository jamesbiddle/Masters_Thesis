%!TEX root = ../thesis.tex
%*******************************************************************************
%*********************************** Fourth Chapter *****************************
%*******************************************************************************

\chapter{Lattice Configurations and the Gluon Propagator} \label{chapter:GluonPropagator}
\ifpdf
    \graphicspath{{Chapter4/Figs/Raster/}{Chapter4/Figs/PDF/}{Chapter4/Figs/}}
\else
    \graphicspath{{Chapter4/Figs/Vector/}{Chapter4/Figs/}}
\fi

Now that we have developed the required background understanding of lattice QCD and the topological objects of interest to this research, we can detail how our calculations are performed. This chapter will first describe how we calculate the primary quantity of interest, the Landau gauge gluon propagator, on the lattice. Then, details of the lattices used for this research will be presented, as well as the choice of momentum variables used in calculations of the gluon propagator. Finall, the choice of data cuts will be described and motivated.

\section{Lattice Definition of the Gluon Propagator}
In a gauge field theory, the position-space propagator, $D_{\mu\nu}(x,y)$, of the gauge boson is the two-point correlation function, which in the case of $SU(3)$ can be interpreted as the probability of a gluon being created at the space-time point $x$, propagating to $y$, and then being annihilated. The propagator therefore serves as a useful measure of the behaviour of gluons as a function of distance; or, correspondingly, as a function of momentum in the momentum-space representation. In this section we detail how the Landau gauge gluon propagator is calculated on the lattice. We begin with the definition of the coordinate space propagator as a two-point correlator~\cite{Zwanziger:1991gz,Cucchieri:1999sz,Langfeld:2001cz}.
\begin{equation}
D^{ab}_{\mu\nu}(x) = \langle A^a_\mu(x) \, A^b_\nu(0)\rangle.
\label{eq:coordGluonProp}
\end{equation}
The propagator in momentum space is simply related by the discrete Fourier transform,
\begin{equation}
D^{ab}_{\mu\nu}(p) = \sum_x e^{-ip\cdot x} \langle A^a_\mu(x) \, A^b_\nu(0) \rangle. 
\end{equation}
Noting that the coordinate space propagator $D^{ab}_{\mu\nu}(x-y)$ only depends on the difference $x-y,$ such that
\begin{equation}
\langle A^a_\mu(x) \, A^b_\nu(0)\rangle = \langle A^a_\mu(x+y) \, A^b_\nu(y)\rangle\, ,
\end{equation}
we can make use of translational invariance to average over the four-dimensional volume to obtain the form for the momentum space propagator.
\begin{align}
D^{ab}_{\mu\nu}(p) &= \frac{1}{V}\sum_{x,y} e^{-ip\cdot x}\langle A^a_\mu(x+y) \, A^b_\nu(y) \rangle \nonumber \\
                &= \frac{1}{V}\sum_{x,y} \langle e^{-ip\cdot (x+y)} A^a_\mu(x+y) \, e^{+ip\cdot y}A^b_\nu(y) \rangle \nonumber \\
                &= \frac{1}{V}\langle A^a_\mu(p) \, A^b_\nu(-p) \rangle. \label{eq:gluPropxtop}
\end{align}

Hence we find that the momentum space gluon propagator on a finite lattice with four-dimensional volume $V$ is given by
%
\begin{equation}
D_{\mu\nu}^{ab}(p) \equiv \frac{1}{V}\left \langle A^a_\mu (p)\,A^b_\nu(-p)\right\rangle \, . \label{eq:gluonProp}
\end{equation}
%
In the continuum, the Landau-gauge momentum-space gluon propagator has the following form~\cite{Leinweber:1998im,Bonnet:2001uh}
%
\begin{equation}
D^{ab}_{\mu\nu}(q) = \left ( \delta_{\mu\nu} - \frac{q_\mu q_\nu}{q^2} \right )\,\delta^{ab}\,D(q^2) \, ,
\end{equation}
%
where $D(q^2)$ is the scalar gluon propagator.  Contracting Gell-Mann index $b$ with $a$ and
Lorentz index $\nu$ with $\mu$ one has
%
\begin{equation}
D^{aa}_{\mu\mu}(q) = (4-1)\,(n_c^2-1)\,D(q^2) \, ,
\end{equation}
%
such that the scalar function can be obtained from the gluon propagator via
%
\begin{equation}
D(q^2) = \frac{1}{3(n_c^2-1)}\,D^{aa}_{\mu\mu}(q) \, ,
\label{eq:scalarProp}
\end{equation}
%
where $n_c = 3$ is the number of colours.

As the lattice gauge links $U_\mu(x)$ naturally reside in the $3\times 3$ fundamental representation of $SU(3),$ we now wish to work in the matrix representation of $A_\mu(x)$, as introduced in Eq.~\ref{eq:CovariantDerivative}. Using the orthogonality relation $\Tr(\lambda_a\lambda_b) = \delta_{ab}$ for the Gell-Mann matrices, it is straightforward to see that
%
\begin{equation}
2\Tr(A_\mu\,A_\mu) = A^a_\mu A^a_\mu\, ,
\end{equation}
%
which can be substituted into equation~\ref{eq:scalarProp} to obtain the final expression for the lattice scalar gluon propagator,
%
\begin{equation}
D(p^2) = \frac{2}{3\,(n_c^2-1)\,V}\big\langle {\rm Tr}\, A_\mu(p)\,A_\mu(-p) \big\rangle \,. \label{eq:scalarProp2}
\end{equation}

To calculate Eq.~\ref{eq:scalarProp2} on the lattice, we need to define $A_\mu(p)$. As defined in Eq.~\ref{eq:GaugePotentialLat}, we make use of the midpoint definition of the coordinate-space gauge potential in terms of the lattice link variables. Once the link variables are fixed to Landau gauge following the procedure described in Sec.~\ref{sec:LandauGauge}, we obtain the momentum-space gauge potential
%
\begin{equation}
A_\mu(p) = \sum_x e^{-ip\cdot(x+\hat{\mu}/2)}\, A_\mu(x+\hat{\mu}/2)\, .
\end{equation}
%
We have now constructed the necessary tools to calculate the Landau gauge scalar gluon propagator within the lattice framework established in Chapter \ref{chapter:LatticeQCD}.

\section{Momentum Variables}
As discussed in Sec.~\ref{sec:Confinement}, it is understood that at QCD is asymptotically free. With this understanding, we expect that at high momentum the Landau gauge gluon propagator will tend towards the Landau gauge photon propagator~\cite{ryder1996quantum}
%
\begin{equation}
D_\gamma(p^2) = \frac{1}{p^2}\, .
\end{equation}
%
\section{Lattice Parameters and Data Cuts}
We calculate the gluon propagator on 100 configurations of a $20^3\times 40$ $SU(3)$ lattice with spacing $a=0.125\,\si{fm}$, as used in Refs.~\cite{Trewartha:2015nna,OMalley:2011aa}. Following the procedure of Ref.~\cite{Bonnet:2001uh,Leinweber:1998im} all results are plotted after a momentum half-cut. The momentum half-cut corresponds to only considering lattice momenta in the range
%
\begin{equation}
p_\mu = \frac{2\pi n_\mu}{a N_\mu},~n_\mu\in \left(-\frac{N_\mu}{4},\,\frac{N_\mu}{4}\right]\, .
\end{equation}
%
This cut limits the positive range of the kinematically corrected $q_\mu$ to
%
\begin{equation}
q_\mu \in \left[0,\, \frac{2\sqrt{21}}{3a}\right]\, .
\end{equation}
%
Furthermore, a cylinder cut of radius $pa=2$ lattice units is performed such that we only consider points within two lattice units of the diagonal, $n(d)=d\,\hat{n} = \frac{d}{2}(1,\,1,\,1,\,1)$. This cut is implemented by considering points satisfying
%
\begin{equation}
|pa|^2\, \sin(\theta_c) \leq 2\, ,
\end{equation}
%
where
%
\begin{equation}
\theta_c = \cos^{-1}\left(\frac{pa \cdot \hat{n}}{|pa|}\right)
\end{equation}
%
This is performed so that all directions are equally sampled, whilst omitting points where one direction dominates the signal. This assists in filtering out high-momentum systematic errors. Finally, we can take advantage of the rotational symmetry of the scalar propagator to perform $Z(3)$ averaging over the Cartesian coordinates. This means that we average over all points with the same Cartesian radius; for example, we would average across the points $(n_x,n_y,n_z)=(2,1,1),\,(1,2,1)$ and $(1,1,2)$. These choices of cuts assist in producing a cleaner signal that accurately represents the behaviour of the continuum propagator.



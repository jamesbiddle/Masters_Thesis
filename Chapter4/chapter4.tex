%!TEX root = ../thesis.tex
%*******************************************************************************
%*********************************** Fourth Chapter *****************************
%*******************************************************************************

\chapter{Landau Gauge Gluon Propagator} \label{chapter:GluonPropagator}
\ifpdf
    \graphicspath{{Chapter4/Figs/Raster/}{Chapter4/Figs/PDF/}{Chapter4/Figs/}}
\else
    \graphicspath{{Chapter4/Figs/Vector/}{Chapter4/Figs/}}
\fi
In a gauge field theory, the position-space propagator, $D_{\mu\nu}(x-y)$, of the gauge boson is the two-point correlation function, which in the case of $SU(3)$ can be interpreted as the probability of a gluon being created at position $x$, propagating to $y$, and then being annihilated. The propagator therefore serves as a useful measure of the behaviour of gluons as a function of distance; or, correspondingly, as a function of momentum in the momentum-space representation. In this chapter we detail how the Landau gauge gluon propagator is calculated on the lattice.
\section{Lattice Definition of the Gluon Propagator}
We begin with the definition of the coordinate space propagator~\cite{Zwanziger:1991gz,Cucchieri:1999sz,Langfeld:2001cz}.
\begin{equation}
D^{ab}_{\mu\nu}(x) = \langle A^a_\mu(x) \, A^b_\nu(0)\rangle.
\label{eq:coordGluonProp}
\end{equation}
The propagator in momentum space is simply related by the discrete Fourier transform,
\begin{equation}
D^{ab}_{\mu\nu}(p) = \sum_x e^{-ip\cdot x} \langle A^a_\mu(x) \, A^b_\nu(0) \rangle. 
\end{equation}
Noting that the coordinate space propagator $D^{ab}_{\mu\nu}(x-y)$ only depends on the difference $x-y,$ such that
\begin{equation}
\langle A^a_\mu(x) \, A^b_\nu(0)\rangle = \langle A^a_\mu(x+y) \, A^b_\nu(y)\rangle\, ,
\end{equation}
we can make use of translational invariance to average over the four-dimensional volume to obtain the form for the momentum space propagator.
\begin{align}
D^{ab}_{\mu\nu}(p) &= \frac{1}{V}\sum_{x,y} e^{-ip\cdot x}\langle A^a_\mu(x+y) \, A^b_\nu(y) \rangle \nonumber \\
                &= \frac{1}{V}\sum_{x,y} \langle e^{-ip\cdot (x+y)} A^a_\mu(x+y) \, e^{+ip\cdot y}A^b_\nu(y) \rangle \nonumber \\
                &= \frac{1}{V}\langle A^a_\mu(p) \, A^b_\nu(-p) \rangle. \label{eq:gluPropxtop}
\end{align}

Hence we find that the momentum space gluon propagator on a finite lattice with four-dimensional volume $V$ is given by
%
\begin{equation}
D_{\mu\nu}^{ab}(p) \equiv \frac{1}{V}\left \langle A^a_\mu (p)\,A^b_\nu(-p)\right\rangle \, . \label{eq:gluonProp}
\end{equation}
%
In the continuum, the Landau-gauge momentum-space gluon propagator has the following form~\cite{Leinweber:1998im,Bonnet:2001uh}
%
\begin{equation}
D^{ab}_{\mu\nu}(q) = \left ( \delta_{\mu\nu} - \frac{q_\mu q_\nu}{q^2} \right )\,\delta^{ab}\,D(q^2) \, ,
\end{equation}
%
where $D(q^2)$ is the scalar gluon propagator.  Contracting Gell-Mann index $b$ with $a$ and
Lorentz index $\nu$ with $\mu$ one has
%
\begin{equation}
D^{aa}_{\mu\mu}(q) = (4-1)\,(n_c^2-1)\,D(q^2) \, ,
\end{equation}
%
such that the scalar function can be obtained from the gluon propagator via
%
\begin{equation}
D(q^2) = \frac{1}{3(n_c^2-1)}\,D^{aa}_{\mu\mu}(q) \, ,
\label{eq:scalarProp}
\end{equation}
%
where $n_c = 3$ is the number of colours.

As the lattice gauge links $U_\mu(x)$ naturally reside in the $3\times 3$ fundamental representation of $SU(3),$ we now wish to work in the matrix representation of $A_\mu(x)$, as introduced in Eq.~\ref{eq:CovariantDerivative}. Using the orthogonality relation $\Tr(\lambda_a\lambda_b) = \delta_{ab}$ for the Gell-Mann matrices, it is straightforward to see that
%
\begin{equation}
2\Tr(A_\mu\,A_\mu) = A^a_\mu A^a_\mu\, ,
\end{equation}
%
which can be substituted into equation~\ref{eq:scalarProp} to obtain the final expression for the lattice scalar gluon propagator,
%
\begin{equation}
D(p^2) = \frac{2}{3\,(n_c^2-1)\,V}\big\langle {\rm Tr}\, A_\mu(p)\,A_\mu(-p) \big\rangle \,. \label{eq:scalarProp2}
\end{equation}

As defined in Eq.~\ref{eq:GaugePotentialLat}, we make use of the midpoint definition of the gauge potential in terms of the lattice link variables. Once the link variables are fixed to Landau gauge following the procedure described in Sec.~\ref{sec:LandauGauge}, we obtain the momentum-space gauge potential
%
\begin{equation}
A_\mu(p) = \sum_x e^{-ip\cdot(x+\hat{\mu}/2)}\, A_\mu(x+\hat{\mu}/2)\, .
\end{equation}

%The gluon fields $U_\mu(x)$ are first gauge-fixed by maximizing an $\mathcal{O}(a^{2})$-improved functional using a Fourier-accelerated algorithm~\cite{Davies:1987vs,Bonnet:1999mj,Roberts:2010cz}. The gauge potential in momentum space is then obtained by taking the discrete Fourier transform, 

\section{Momentum Variables}

\section{Lattice Parameters and Data Cuts}


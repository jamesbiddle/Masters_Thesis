%!TEX root = ../thesis.tex
%*******************************************************************************
%****************************** Second Chapter *********************************
%*******************************************************************************

\chapter{Lattice QCD}\label{chapter:LatticeQCD}

\ifpdf
    \graphicspath{{Chapter2/Figs/Raster/}{Chapter2/Figs/PDF/}{Chapter2/Figs/}}
\else
    \graphicspath{{Chapter2/Figs/Vector/}{Chapter2/Figs/}}
\fi
Currently, lattice QCD represents the only technique able to perform accurate low-energy QCD calculations from first principles. The lattice prescription allows for the explicit calculation of path integrals present in continuum QCD, at the cost of introducing finite-spacing errors that must be systematically accounted for. In this chapter we will discuss the behaviour of QCD in the continuum, and demonstrate how the transition can be made to a finite set of coordinates on a lattice. We will then briefly detail how this formulation can be used to generate simulations of the QCD vacuum. Finally, we use this framework to describe how our generated configurations can be fixed to Landau gauge, one of the two gauge choices used in this research.

\section{QCD in the Continuum}
\subsection{Quarks and Gauge Invariance}
QCD is the gauge field theory that describes the interactions of quarks and gluons. Like all gauge theories, it has an internal symmetry group under which the Lagrangian is invariant. In the case of QCD there are three quark colours, which leads to the symmetry group being $SU(3)$, the group of $3\times 3$ unitary matrices of determinant 1. Note that this description of $SU(3)$ is only true in the fundamental representation, however it is this representation that the quarks inhabit and is therefore a useful and intuitive way to initially consider the group. We can observe this $SU(3)$ symmetry by inspecting the free quark Lagrangian
%
\begin{equation}
\mathcal{L}_0 = \bar{\psi}(x)\,(i\slashed{\partial}-m)\,\psi(x)\,.
\label{eq:GlobalQuarkLagrangian}
\end{equation}
%
where $\psi(x)$ and $\bar{\psi}(x)$ contain the three quark and anti-quark fields respectively, $m$ is the quark mass and $\slashed{\partial}=\partial_\mu\,\gamma^\mu$. We make use of the Dirac representation for the gamma matrices, given in Appendix~\ref{app:GellMann}. If we apply an $SU(3)$ transformation $\Omega$ to the three colour indices of the quark and anti-quark fields such that
%
\begin{align}
\psi(x)&\rightarrow\Omega\,\psi(x)\\
\bar{\psi}(x)&\rightarrow \bar{\psi}(x)\,\Omega^\dagger\, ,
\end{align}
%
we see that
%
\begin{align}
\mathcal{L}_0\rightarrow\mathcal{L'}_0&=\bar{\psi}(x)\,\Omega^\dag\,(i\slashed{\partial}-m)\,\Omega\,\psi(x)\nonumber\\
&= \bar{\psi}(x)\,(i\slashed{\partial}-m)\,\Omega^\dag\,\Omega\,\psi(x)\nonumber\\
&= \bar{\psi}(x)\,(i\slashed{\partial}-m)\,\psi(x)\nonumber\\
&= \mathcal{L}_0\, ,
\end{align}
%
where we have made use of the unitarity property, $\Omega\,\Omega^\dag = I$. If this symmetry were all we required then $\mathcal{L}_0$ would be our Lagrangian and our theory would be pleasantly simple. However, we find that we need our gauge symmetry to be \textit{local}; that is, we demand that our gauge transformation itself be a function of $x$~\cite{peskin2018introduction}. In this case, we find that the derivative in Eq.~\eqref{eq:GlobalQuarkLagrangian} acting on the gauge transformation results in a loss of $SU(3)$ symmetry. We can write an arbitrary local $SU(3)$ gauge transformation as an exponential of the eight traceless, Hermitian group generators $\lambda_a$, known as the Gell-Mann matrices (see Appendix~\ref{app:GellMann} for their values), such that 
%
\begin{equation}
\Omega(x)=\exp\left(i\omega^a(x)\,\frac{\lambda_a}{2}\right)\, .
\label{eq:LocalGaugeTransformation}
\end{equation}
%
Note that we make use of the summation convention to imply a sum over the repeated indices. In this form, the spatial dependence is encapsulated entirely in the eight parameters, $\omega^a(x)$.  Using this form for $\Omega(x)$, we find that under a gauge transformation the Lagrangian is now
%
\begin{align}
\mathcal{L}_0\rightarrow\mathcal{L^\prime}_0&=\bar{\psi}(x)\,\Omega^\dag(x)\,(i\slashed{\partial}-m)\,\Omega(x)\,\psi(x)\nonumber\\
&=\bar{\psi}(x)\,\Omega^\dag(x)\left[-\frac{\lambda_a}{2}\,(\slashed{\partial}\,\omega^a(x))\,\Omega(x)\,\psi(x)+ i\,\Omega(x)\,(\slashed{\partial}\,\psi(x))-m\,\Omega(x)\,\psi(x)\right]\nonumber\\
&= \mathcal{L}_0-\bar{\psi}(x)\,\Omega^\dag(x)\,\frac{\lambda_a}{2}\,(\slashed{\partial}\,\omega^a(x))\,\Omega(x)\,\psi(x)\, .
\label{eq:LocalTrans}
\end{align}
%
It is apparent then that gauge invariance is lost under a local gauge transformation. To restore gauge invariance, we introduce the notion of the gauge-covariant derivative
%
\begin{equation}
D_\mu = \partial_\mu + ig A_\mu(x)\, ,
\label{eq:CovariantDerivative}
\end{equation}
%
where $A_\mu(x)=A_\mu^a(x)\,\frac{\lambda_a}{2}$ encapsulates the eight new `gauge potentials' and $g$ is the strong coupling constant. As the gauge potentials are a linear combination of $\lambda_a$, they belong not to the group $SU(3)$, but to the Lie algebra $\mathfrak{su}(3)$. In the context of QCD, these gauge potentials are also known as the gluon field. For the sake of cleanliness, we will stop explicitly writing the dependence of our fields and gauge transformations on $x$ from here on, but it should be remembered that all gauge transformations are local unless explicitly stated otherwise.\\

Making the substitution $\partial_\mu\rightarrow D_\mu$, we obtain the new Lagrangian
%
\begin{equation}
\mathcal{L}_\text{quark} = \bar{\psi}\,(i\slashed{D}-m)\,\psi\,.
\end{equation}
%
This substitution introduces a new interaction term into the Lagrangian that gives rise to an interaction between our quark and gauge fields.
%
\begin{equation}
\mathcal{L}_\text{int} = -g\,\bar{\psi}\,A_\mu\,\psi
\label{eq:InteractingLagrangian}
\end{equation}
%
To preserve the gauge invariance of the Lagrangian, we need the gauge transformation property of Eq.~\eqref{eq:InteractingLagrangian} to counteract the last term of Eq.~\eqref{eq:LocalTrans}. Hence we require that
%
\begin{align}
-g\,\bar{\psi}\,A_\mu\,\psi \rightarrow -g\,\bar{\psi}\,A_\mu\,\psi + \bar{\psi}\,\Omega^\dag\,\frac{\lambda_a}{2}(\partial_\mu\,\omega^a)\,\Omega\,\psi\, .
\end{align}
%
Making use of the transformation properties of $\psi$ and $\bar{\psi}$, this implies that
%
\begin{equation}
A_\mu\rightarrow \Omega\,A_\mu\,\Omega^\dag + \frac{i}{g}\,(\partial_\mu\,\Omega)\,\Omega^\dag\, . 
\label{eq:GaugePotentialTrans}
\end{equation}
%
This transformation property can also be expressed in terms of the covariant derivative. Doing so, we find that
%
\begin{align}
D_\mu\,\psi \rightarrow &\left(\partial_\mu +ig\,\Omega\,A_\mu\,\Omega^\dag - (\partial_\mu\,\Omega)\,\Omega^\dag\right)\,\Omega\,\psi\nonumber\\
&= (\partial_\mu\,\Omega)\,\psi + \Omega\,(\partial_\mu\,\psi) + ig\,\Omega\,A_\mu\,\psi - (\partial_\mu\,\Omega)\psi\nonumber\\
&=\Omega\,D_\mu\,\psi\, .\label{eq:CovariantTransformation}
\end{align}
%
And therefore
%
\begin{equation}
D_{\mu}\rightarrow\Omega\,D_\mu \Omega^\dag
\end{equation}
%
Eq.~\eqref{eq:CovariantTransformation} tells us that the covariant derivative of a quark field transforms in the same way as the quark field itself. This implies that the covariant derivative can be understood as a connection between two points that may have a different underlying gauge. For example, if we consider an infinitesimal translation in the quark field
%
\begin{equation}
d\psi(x) = \psi(x+dx)-\psi(x)\, ,
\end{equation} 
%
we note that the gauge at the point $x$ and at $x+dx$ in general will differ. It therefore does not make sense to compare the field values through the usual understanding of the derivative, as this ignores the change in local gauge. Instead, the covariant derivative accounts for this underlying gauge structure, `transporting' the field from one position to another. This is entirely analogous to the covariant derivative present in general relativity, however in this case the transport occurs over an internal gauge manifold, rather than an external curved space-time.\\

\subsection{Field Strength Tensor}
Using local gauge invariance as a guide, we can now seek other gauge invariant terms to insert into the Lagrangian. We define the gluon field strength tensor to be

%If we consider the commutator of the covariant derivative, we have
%
%\begin{align*}
%[D_\mu,\, D_\nu]\rightarrow &[\Omega\,D_\mu\,\Omega^\dag,\,\Omega\,D_\nu\,\Omega^\dag]\\
%&= \Omega\,D_\mu\, D_\nu\,\Omega^\dag - \Omega\,D_\nu\, D_\mu\,\Omega^\dag\\
%&= \Omega\,[D_\mu,\,D_\nu]\,\Omega^\dag\, .
%\end{align*}
%

%
\begin{equation}
F_{\mu\nu}=\partial_\mu A_\nu - \partial_\nu A_\mu + ig[A_\mu,\,A_\nu]\, .
\label{eq:FieldStrengthTensor}
\end{equation}
%
Alternatively, $F_{\mu\nu}$ may also be written
%
\begin{align}
F_{\mu\nu} = -\frac{i}{g}\,[D_\mu,\, D_\nu]\, .
\end{align}
%
By making use of the gauge transformation property of $A_\mu$, given in Eq.~\eqref{eq:GaugePotentialTrans}, we find that the field strength tensor transforms as
%
\begin{equation}
F_{\mu\nu}\rightarrow \Omega\,F_{\mu\nu}\, \Omega^\dagger\, .
\end{equation}
%
The proof of this is given in Appendix~\ref{app:GTF}. To obtain a gauge invariant quantity, we take the trace of the contracted field strength tensor. This allows us to make use of the cyclic property of the trace to obtain
%
\begin{align}
\Tr(F_{\mu\nu}F^{\mu\nu})\rightarrow &\Tr\left(\Omega\,F_{\mu\nu}\, \Omega^\dagger\,\Omega\,F^{\mu\nu}\, \Omega^\dagger\right)\nonumber\\
&=\Tr\left(\Omega^\dagger \, \Omega\,F_{\mu\nu}\,F^{\mu\nu}\right)\nonumber\\
&=\Tr(F_{\mu\nu}F^{\mu\nu})
\end{align}
Thus we define the full gauge invariant QCD Lagrangian to be
%
\begin{equation}
\mathcal{L_{\text{QCD}}}=\bar{\psi}(x)\,(i\slashed{D}-m)\,\psi(x)-\frac{1}{2}\,\Tr(F_{\mu\nu}(x)\,F^{\mu\nu}(x))\, .
\label{eq:QCDLagrangian}
\end{equation}\\

This gluon term is not the only gauge invariant quantity we could construct; for example, $\bar{\psi}\,\psi\,\bar{\psi}\,\psi$ is clearly gauge invariant. However, it turns out that there is a further condition that must be satisfied by each term in the Lagrangian; each term must be \textit{renormalisable}~\cite{peskin2018introduction}. A complete discussion of renormalisation is unnecessary for this work, but renormalisability can be quickly summarised by looking at the dimensionality of each term in the Lagrangian. The Lagrangian must have units of $(\text{Energy})^4$, which in natural units is $(\text{mass})^4$,  hereafter referred to as just dimension $D=4$. We therefore require that each term and its accompanying coupling constant give the same dimensionality. The fermion field has dimension $\frac{3}{2}$, the gauge potential has dimension 1 and $\partial_\mu$ has dimension 1. Thus, the terms present in Eq.~\eqref{eq:QCDLagrangian} have dimension
%
\begin{align}
\text{D}[\bar{\psi}(x)\,\gamma^\mu\,\partial_\mu\,\psi(x)]&=\frac{3}{2}+1+\frac{3}{2}=4\\
\text{D}[\bar{\psi}(x)\,\gamma^\mu\,A_\mu\,\psi(x)]&=\frac{3}{2}+1+\frac{3}{2}=4\\
\text{D}[m\bar{\psi}(x)\,\psi(x)]&=1+\frac{3}{2}+\frac{3}{2}=4\\
\text{D}[F_{\mu\nu}\,F^{\mu\nu}] &= 2+2=4\, ,
\end{align}
%
as required. This also tells us that the coupling constant $g$ is dimensionless. If a new gauge invariant term $h\bar{\psi}\,\psi\,\bar{\psi}\,\psi$ with coupling constant $h$ is introduced then by the above rules we would require that $h$ have dimension $-2$. It turns out that if the dimensionality of the coupling constant is less than 0 then the term in non-renormalisable. This means that integrals involving this new term will diverge in such a way that they cannot be systematically be made finite through the use of a renormalisation scheme, and hence they cannot form part of any physical theory. By applying the requirements of gauge invariance and renormalisability, it is apparent that Eq.~\eqref{eq:QCDLagrangian} is the full QCD Lagrangian.

\subsection{Pure Gauge Action}
For the purpose of this research, we are interested in the behaviour of gluons in the absence of any quarks, and as such we need to develop a description of pure gauge fields. In the continuum, a pure gauge field has the Lagrangian~\cite{ryder1996quantum}
%
\begin{equation}
\mathcal{L}_{\text{gluon}}=\frac{1}{2}\Tr(F_{\mu\nu}\,F^{\mu\nu})\, ,
\label{eq:GaugeLagrangian}
\end{equation}
%
which we observe to be the last term in Eq.~\eqref{eq:QCDLagrangian}. This Lagrangian has the corresponding action
%
\begin{equation}
\mathcal{S}=\int~d^4x~\mathcal{L_\text{gluon}}\, .
\label{eq:QCDAction}
\end{equation}
%
When considering the path integral formulation of a gauge field theory, integrals such as the generating functional,
%
\begin{equation}
\mathcal{Z} =\int \mathcal{D} A_\mu \exp\left(i\,\mathcal{S}\,[A_\mu]\right),
\label{eq:GeneratingFunctional}
\end{equation}
%
and others of a similar form appear frequently. This integral closely resembles the partition function found in statistical mechanics, $\mathcal{Z}_{\text{classical}}=\int d^3x\,d^3p\,\exp\left(-\beta\,H(x,p)\right)$, with the notable exception of the factor of $i$ in the exponential. From the statistical mechanics perspective, the exponential in Eq.~\eqref{eq:GeneratingFunctional} is a probability weighting for a given gauge potential. However, unlike the classical case, the factor of $i$ in Eq.~\eqref{eq:GeneratingFunctional} results in an oscillatory weighting, rendering numerical simulations untenable. To ensure that the weight factor is purely real, it is necessary to perform a Wick rotation to Euclidean space~\cite{Schafer:1996wv,Wilson:1974sk} such that
\begin{equation}
t \rightarrow -it \qquad A^0 \rightarrow iA^0\, .
\end{equation}
%
This has the result of changing the action such that
%
\begin{equation}
i\mathcal{S}_\text{Minkowski} \rightarrow -\mathcal{S}_\text{Euclidean}\, ,
\end{equation} 
so that the generating functional now becomes 
%
\begin{equation}
\mathcal{Z}=\int \mathcal{D} A_\mu \exp\left(-\mathcal{S}_E\,[A_\mu]\right).
\end{equation}
%
This enables us to now truly consider the generating functional to be a probability weighting for a given configuration.\\

In Euclidean space, we can make use of the generating functional to write the expectation value of an arbitrary operator $Q[A_\mu]$ as~\cite{Luscher:1984xn}
%
\begin{equation}
\langle Q \rangle = \frac{1}{\mathcal{Z}}\int \mathcal{D} A_\mu\, Q[A_\mu]\, \exp\left(-\mathcal{S}_E\,[A_\mu]\right)\, .
\end{equation}
%
This definition of the expectation value, whilst potentially difficult or even impossible to calculate analytically, has an intuitive interpretation. To calculate the expectation value of some operator, we integrate over every possible configuration of $A_\mu(x)$, weighted by the action of that configuration. In the case where the coupling constant $g$ of the theory is sufficiently small, as is the case for QED or high-energy QCD, it is possible to expand $\exp\left(-\mathcal{S}_E\,[A_\mu]\right)$ in terms of the coupling constant, leading to a perturbative expansion. Alternatively, if the only relevant configurations in the theory are those near the classical action satisfying $\frac{\delta S[A_\mu]}{\delta A_\mu}=0$, then the action can be expanded around the classical solution. However, in the case of low-energy QCD, both of these approximations are invalid, and as such it becomes essential to sample possible configurations of $A_\mu(x)$ and generate a representative finite subset that can be used to approximate the continuum expectation value. Obtaining this subset on which we can perform calculations is one of the key aims of lattice QCD.

\section{Lattice Discretisation}\label{sec:LatticeDiscretisation}
Using the continuum understanding developed in the previous section, we can now consider discretising space-time into a finite lattice. The lattice is a hypercube with $N_s$ lattice sites in the spacial directions and $N_t$ sites in the time direction. Each lattice site is separated by a spacing $a$, resulting in a total lattice volume $V=(N_s\,a)^3\times N_t\,a$. A two dimensional example of a discrete lattice with spacing $a$ is shown in Fig.~\ref{fig:LatticeExample}. The lattice notation $\hat{\mu}$ is used to denote the unit vector in the $\mu$ direction; for example, $\hat{y} = (0,0,1,0)$. We also must impose boundary conditions for the lattice; in this work we utilise periodic boundary conditions such that $x+N_\mu a\hat{\mu}=x$.\\
%
\begin{figure}[ht]
\centering
\begin{tikzpicture}
\tkzDefPoint(0,0){A}
\tkzDefPoint(3,0){B}
\tkzDefPoint(0,3){C}
\tkzDefPoint(3,3){D}

\tkzDefShiftPoint[A](0.15,-0.3){A'}
\tkzDefShiftPoint[A](-0.3,0.3){A''}
\tkzDefShiftPoint[A](0.3,0.3){A'''}
\tkzDefShiftPoint[B](-0.15,-0.3){B'}
\tkzDefShiftPoint[B](-0.3,0.3){B''}
\tkzDefShiftPoint[C](-0.3,-0.3){C'}
\tkzDefShiftPoint[C](0.3,-0.3){C''}
\tkzDefShiftPoint[D](-0.3,-0.3){D'}

\begin{scope}[very thick,decoration={
    markings,
    mark=at position 0.55 with {\arrow[scale=2.5]{stealth}}}
    ] 
\tkzDrawSegments[postaction={decorate}](A,C);
\end{scope}

\draw [step=3cm, line width=2pt] (-3-0.0001,-3-0.0001) grid (3,3);
\draw plot[mark=*,mark size = 3pt,mark options={color=black!30!red}] coordinates {(-3,-3)(0,-3)(0,0)(-3,0)(-3,3)(0,3)(3,3)(3,0)(3,-3)};
\node at (-0.25,-0.25) {$x$};
\node at (3.9,0) {$x+a\hat{\mu}$};
\node at (-3.9,0) {$x-a\hat{\mu}$};
\node at (0.5,3.4) {$x+a\hat{\nu}$};
\node at (0.5,-3.4) {$x-a\hat{\nu}$};

\begin{scope}[very thick,decoration={
    markings,
    mark=at position 0.55 with {\arrow[scale=2]{stealth}}}
    ] 
\tkzDrawSegments[postaction={decorate}](A''',B'');
\tkzDrawSegments[postaction={decorate}](B'',D');
\tkzDrawSegments[postaction={decorate}](D',C'');
\tkzDrawSegments[postaction={decorate}](C'',A''');
\end{scope}

\tkzDrawSegments[thick, <->, >=stealth](A',B') 

\tkzLabelSegment[below=-0.25,fill=white,inner sep=5pt](A',B'){$a$}
\tkzLabelSegment[left=0.15,fill=white,inner sep=5pt](A,C){$U_\nu(x)$}
\tkzLabelSegment[above=0.75](A''',B''){$P_{\mu\nu}(x)$}
\end{tikzpicture}
\caption[An example of a 2D lattice with lattice spacing $a$.]{\label{fig:LatticeExample} An example of a 2D lattice with lattice spacing $a$. From site $x$ we define $x+a\hat{\mu}$ to refer to the next lattice site in the $\hat{\mu}$ direction. The gauge links $U_\mu(x)$ (see Eq.~\eqref{eq:GaugeLink}) are defined on the links between sites. The plaquette $P_{\mu\nu}(x)$ (see Eq.~\eqref{eq:Plaquette}) is the product of the four gauge links around a $1\times 1$ loop.}
\end{figure}
%

When space-time is discretised, it becomes necessary to consider derivatives as finite differences and integrals as finite sums, such that.
\begin{align}
\partial_\mu\,f(x)&\rightarrow \frac{f(x+a\hat{\mu})-f(x-a\hat{\mu})}{2a}\\
\int d^4x~f(x) &\rightarrow a^4\sum_x \,f(x)\, .
\end{align}
For example, we can construct the lattice form of Eq.~\eqref{eq:FieldStrengthTensor} as
%
\begin{align}
F_{\text{Lat}}^{\mu\nu}(x) = &\frac{A_\nu(x+a\hat{\mu})-A_\nu(x-a\hat{\mu})}{2a}-\frac{A_\mu(x+a\hat{\nu})-A_\mu(x-a\hat{\nu})}{2a}\nonumber\\
&+ig[A_\mu(x),\,A_\nu(x)]\, .
\label{eq:DiscreteFST}
\end{align}
%
The notation $A_\nu(x+a\hat{\mu})$ denotes the field $A_\nu$ located at the site one lattice spacing in the $\hat{\mu}$ direction from $x$. We could continue to reformulate our lattice theory by imposing this method of discretisation, and indeed this is historically how the lattice framework was constructed~\cite{Wilson:1974sk}. However, it is useful to instead formulate our lattice theory in terms of gauge {\it links}. Analogous to how we introduced the covariant derivative to compensate for the fact that the quark field at infinitesimally different points in space has a different underlying gauge, we now want to have a mechanism for comparing gluon fields at some finite separation. This requires us to solve the parallel transport equation of our gauge field~\cite{peskin2018introduction}
%
\begin{equation}
\frac{dx^\mu(t)}{dt}\,D_\mu \,U(x(t),y)=0\, ,
\label{eq:ParallelTransport}
\end{equation}
%
where $U(x(t),y)$ is an $SU(3)$ element and $x(t)$ is some path parametrised by $t\,\in\,[0,1]$ satisfying $x(0)=y$. We further require that $U(x(0),y)=I$, as the parallel transport for a fixed point is trivial. We can now make use of the explicit parametrisation of the path between two adjacent lattice sites, $x^\mu(t;\nu) = y^\mu+a\,t\,\delta_\nu^\mu$, where $y^\mu$ is a fixed position and $\nu$ is the direction we are transporting the field. Substituting this parametrisation into Eq.~\eqref{eq:ParallelTransport} we have
\begin{align}
&a\, \delta_\nu^\mu\, (\partial_\mu + igA_\mu)\,U(x(t;\nu),y)=0\nonumber\\
&a\, \partial_\nu\, U(x(t;\nu),y) = -iag\, A_\nu\,U(x(t;\nu),y)\nonumber\\
&\frac{\partial}{\partial t}\, U(x(t;\nu),y) = -iag\,A_\nu\, U(x(t;\nu),y)\, .
\label{eq:PathOrderedDE}
\end{align}
For a non-Abelian field, Eq.~\eqref{eq:PathOrderedDE} is precisely the differential equation solved by the path-ordered exponential, known as the Wilson line
%
\begin{equation}
U(x(t;\nu),y) = \mathcal{P}\exp\left(-iag\int_0^t \,dt^\prime \,A_\nu(x(t^\prime;\nu))\right)
\end{equation}
%
Hence, for each direction $\hat{\mu}$, we define the gauge links between adjacent lattice sites to be
%
\begin{equation}
U_\mu(x) = \mathcal{P}\exp\left(-iag\int_0^1 \,dt \,A_\mu(x + at\hat{\mu})\right)\, .
\label{eq:GaugeLink}
\end{equation}
%
From this definition we also see that we can write the gauge link in the opposite direction, i.e. from $x+a\hat{\mu}$ to $x$, as
%
\begin{align}
\mathcal{P}\exp\left(-iag\int^0_1 \,dt\,A_\mu(x + at\hat{\mu})\right) &= \mathcal{P}\exp\left(+iag\int_0^1 \,dt\,A_\mu(x + at\hat{\mu})\right)\nonumber\\
&=U^\dag_\mu(x)\, .
\end{align}
%
These gauge links have the simple gauge transformation property~\cite{Lepage:1998dt} (see Appendix~\ref{app:WilsonLineGT})
%
\begin{equation}
U_\mu(x)\rightarrow \Omega(x)\,U_\mu(x)\,\Omega^\dag(x+a\hat{\mu})\, .
\label{eq:LinkTransformation}
\end{equation}
%
Making use of this gauge transformation property, we can construct gauge invariant Wilson loops by taking the trace of the product of the $U_\mu$'s around a closed loop. These Wilson loops form an essential building block of the lattice action, and  appear in later chapters as quantity of interest in their own right. The simplest such loop, the $1\times 1$ square, is called the \textit{plaquette}, and is defined as
\begin{equation}
P_{\mu\nu}(x) = U_\mu(x)\,U_\nu(x+a\hat{\mu})\, U_\mu^\dag(x+a\hat{\nu})\, U_\nu^\dag(x)\, .
\label{eq:Plaquette}
\end{equation}
Calculating the Wilson loop by taking the trace of the plaquette we see that, by the cyclic property of the trace, the Wilson loop is gauge invariant
\begin{align}
\Tr\left(P_{\mu\nu}(x)\right)\rightarrow& \Tr \left(\Omega(x)\,U_\mu(x)\Omega^\dag(x+a\hat{\mu})\,\Omega(x+a\hat{\mu})\,U_\nu(x+a\hat{\mu})\,\Omega^\dag(x+a\hat{\mu}+a\hat{\nu})\right.\nonumber\\
&~~~~~\left.\Omega(x+a\hat{\mu}+a\hat{\nu})\,U_\mu^\dag(x+a\hat{\nu})\,\Omega^\dag(x+a\hat{\nu})\,\Omega(x+a\hat{\mu})\, U_\nu^\dag(x)\,\Omega^\dag(x)\right)\nonumber\\
=&\Tr\left(P_{\mu\nu}(x)\right)\, .
\end{align}
Both the gauge links and the plaquette are also visualised in Fig.~\ref{fig:LatticeExample}.\\

We now return to the lattice formulation of QCD, making use of the gauge links to define our quantities of interest. Firstly, we approximate our gauge links on the lattice by using a midpoint definition, such that
\begin{equation}
U_\mu^\text{lat}(x) = \exp\left(-iag\, A_\mu\left(x+\frac{a}{2}\hat{\mu}\right)\right)\, .
\label{eq:GaugeLinkLat}
\end{equation}
From this definition, we can also recover the midpoint gauge potential~\cite{Leinweber:1998im,Alles:1996ka}
\begin{equation}
A_\mu\left(x+\frac{a}{2}\hat{\mu}\right) = \frac{i}{2ag}\left(U_\mu(x) - U_\mu^\dag(x)\right) - \frac{i}{6ag}\Tr\left(U_\mu(x) - U_\mu^\dag(x)\right)I + \mathcal{O}(a^2)\, .
\label{eq:GaugePotentialLat}
\end{equation}
We then note that we can write $F_{\mu\nu}$ in terms of the plaquette by Taylor expanding Eq.~\eqref{eq:Plaquette} (see Appendix \ref{app:TEPlaquette}) to obtain~\cite{Gupta:1997nd}
%
\begin{equation}
P_{\mu\nu} = I-ia^2g\, F_{\mu\nu} - \frac{a^4 g^2}{2}F^2_{\mu\nu} +\mathcal{O}(a^6)\, ,
\label{eq:PlaquetteExpansion}
\end{equation} 
%
and hence to $\mathcal{O}(a^2)$
%
\begin{align}
\frac{a^4}{2}\Tr\left(F_{\mu\nu}F^{\mu\nu}\right) = \sum_{\mu,\,\nu}\frac{1}{g^2}\Tr\left(I-\frac{1}{2}\left(P_{\mu\nu}+P_{\mu\nu}^\dag\right)\right)\, .
\label{eq:FieldStrengthPlaquette}
\end{align}
%
We have now arrived at a definition of the contracted field strength tensor that can be used to define our lattice action. We can make a further simplification by noting that because $P_{\mu\nu}=P_{\nu\mu}^\dagger$, $\Re(P_{\mu\nu}) = \Re(P_{\nu\mu})$ and therefore we only need to sum over the 6 plaquettes for which $\mu<\nu$, so long as we introduce a factor of $2$. This gives us the definition of the Wilson action, 
%
\begin{equation}
\mathcal{S}_\text{W} = \beta\sum_x\,\sum_{\mu<\nu} \frac{1}{3}\Tr\left(I-\frac{1}{2}\left(P_{\mu\nu}+P_{\mu\nu}^\dag\right)\right)\, ,
\label{eq:WilsonAction}
\end{equation}
%
where $\beta = \frac{6}{g^2}$ is the lattice coupling constant. To remove higher order errors from the lattice action, it is possible to take into account terms containing larger Wilson loops, following procedure similar to the one outlined above~\cite{Alford:1995hw,Symanzik:1983dc,Symanzik:1983gh}.\\

For the purpose of this work, the gauge fields were generated using the $\mathcal{O}(a^2)$-improved L\"uscher-Weisz action~\cite{Luscher:1984xn}, 
%
\begin{align}
\mathcal{S} _ { LW } = &\sum_x \left[ \frac { 5 \beta } { 9 } \sum _ { \mu < \nu } \operatorname { Tr } \left\{ 1 - \frac { 1 } { 2 } \left( P _ { \mu \nu } + P _ { \mu \nu } ^ { \dagger } \right) \right\}\right. \nonumber\\
& \left.- \frac { \beta } { 36 u _ { 0 } ^ { 2 } } \sum _ { \text { rect } } \operatorname { Tr } \left\{ 1 - \frac { 1 } { 2 } \left( R _ { \mu \nu } + R _ { \mu \nu } ^ { \dagger } \right) \right\}\right]\, ,
\end{align}
%
where
\begin{equation}
u_0 = \left(\frac{1}{3}\operatorname{ Re } \Tr\langle P_{\mu\nu} \rangle\right)^{\frac{1}{4}}\, ,
\end{equation}
and $R_{\mu\nu}$ is the $2\times 1 + 1\times 2$ rectangular Wilson loop, defined similarly to the plaquette 
\begin{align}
 R _ { \mu \nu } ( x ) = \, &  U_\mu( x )\, U_\nu( x + \hat { \mu } )\, U_\nu( x + \hat { \nu } + \hat { \mu } )\, U_\mu^\dagger ( x + 2 \hat { \nu } )\, U _\nu^\dagger( x + \hat { \nu } )\, U_\nu^\dagger( x )\nonumber\\
&+ U_\mu ( x )\, U_\mu ( x + \hat{\mu} )\, U_\nu( x + 2 \hat { \mu } )\, U_\mu^\dagger ( x + \hat { \mu } + \hat { \nu } )\, U_\mu^\dagger ( x + \hat { \nu } )\, U_\nu^\dagger( x ) \, .
\end{align}
The presence of the `tadpole' improvement factor $u_0$ is necessary to ensure the perturbatively defined coefficient is accurate~\cite{Lepage:1992xa}. This choice of action provides reduced errors in comparison to the Wilson action.\\

This lattice framework provides the tools necessary to explicitly calculate quantities of interest from a first-principles standpoint. Firstly, the gauge links are generated by Markov-chain Monte Carlo methods, using $\exp\left(-\mathcal{S}\right)$ as a probability weighting in the Metropolis accept/reject for a given configuration. Once these configurations are generated, gauge fixing can be performed (Sec.~\ref{sec:LandauGauge}, \ref{sec:MCG}), and quantities of interest such as the gluon propagator (Chapter~\ref{chapter:GluonPropagator}) can be obtained.

\section{Gauge Fixing}
The choice of gauge is crucial when performing calculations of quantities which are gauge dependent. There are two choices of gauge relevant to this study: Landau gauge and maximal centre gauge. Maximal centre gauge is best explored in the context of centre vortices, and will therefore be detailed in Chapter~\ref{sec:MCG}, however the Landau gauge fixing condition provides a good introduction to the gauge-fixing procedure, and as such will be described here.

\subsection{Landau Gauge}\label{sec:LandauGauge}

In the continuum, Landau gauge corresponds to imposing the condition
\begin{equation}
\partial_\mu A^\mu = 0\, .
\label{eq:LandauGaugeCont}
\end{equation}
%
On the lattice, we can approximate this condition by imposing
\begin{equation}
\Delta(x) = \sum _ { \mu } A _ { \mu } \left( x + \frac{a}{2}\hat { \mu } \right) - A _ { \mu } \left( x-\frac{a}{2}\hat { \mu } \right) = 0\, .
\label{eq:LandauGaugeLat}
\end{equation}
Here the fact that we have defined the lattice gauge potential to be at the midpoint of the link produces an improved continuum limit when we consider Eq.~\eqref{eq:LandauGaugeLat} in momentum space~\cite{Alles:1996ka}. The Landau gauge condition is imposed on the lattice by finding extrema of the $\mathcal{O}(a^2)$-improved functional~\cite{Bonnet:1999mj}
%
\begin{equation}
\mathcal{F} =  \frac{4}{3}\mathcal{F}_1 - \frac{1}{12u_0}\mathcal{F}_2\, ,
\label{eq:LGFunctional}
\end{equation}
%
where
%
\begin{align}
\mathcal{F}_1 &= \sum _ { \mu , x } \frac { 1 } { 2 } \operatorname { Tr } \left\{ U _ { \mu } ^ { \Omega } ( x ) + U _ { \mu } ^ { \Omega } ( x ) ^ { \dagger } \right\}\\
\mathcal{F}_2 &= \sum _ { \mu , x } \frac { 1 } { 2 } \operatorname { Tr } \left\{ U _ { \mu } ^ { \Omega } ( x ) \,U _ { \mu } ^ { \Omega } ( x + a\hat { \mu } ) + U _ { \mu } ^ { \Omega } ( x + a\hat { \mu } )^\dagger\, U _ { \mu } ^ { \Omega } ( x )^\dagger  \right\}\, .
\end{align}
%
We explicitly write $U^\Omega_\mu$ to emphasise that we are considering gauge links under an as yet unknown gauge transformation $\Omega$. It becomes apparent why we seek the extrema of this particular functional when we take the functional derivative with respect to the free parameters of the gauge transformation , $\omega^a(x)$ (see Eq.~\eqref{eq:LocalGaugeTransformation}).
%
\begin{equation}
\frac { \delta \left\{ \frac { 4 } { 3 } \mathcal { F } _ { 1 } - \frac { 1 } { 12 u _ { 0 } } \mathcal { F } _ { 2 } \right\} } { \delta \omega ^ { a } ( x ) } = g a ^ { 2 } \sum _ { \mu } \operatorname { Tr } \left\{ \left[ \partial _ { \mu } A _ { \mu } ( x ) - \frac { 4 } { 360 } a ^ { 4 } \partial _ { \mu } ^ { 5 } A _ { \mu } ( x ) + \mathcal { O } \left( a ^ { 6 } \right) \right] \frac{\lambda^a}{2} \right\} + \mathcal { O } \left( g ^ { 3 } a ^ { 4 } \right)\, .
\label{eq:LGFunctionalDeriv}
\end{equation}
%
If Eq.~\eqref{eq:LGFunctionalDeriv} is at an extrema, then 
%
\begin{equation}
\sum_\mu \partial_\mu A_\mu(x) = \sum_\mu \frac{4}{360}a^4 \partial_\mu^5\,A_\mu(x) + \mathcal{O}(a^6)+\mathcal{O}(g^3a^4)\, .
\end{equation}
%
Hence up to errors of order $\mathcal{O}(a^4)$, finding the extrema of Eq.~\eqref{eq:LGFunctionalDeriv} is equivalent to satisfying the continuum Landau gauge condition given in Eq.~\eqref{eq:LandauGaugeCont}. This Landau gauge fixing method gives an example of how a gauge choice can be implemented on a discrete lattice such that it approximates the continuum condition. This in turn enables us to use the continuum Landau gauge definition of the gluon propagator as described in Chapter~\ref{chapter:GluonPropagator}, which forms a vital component of this research.

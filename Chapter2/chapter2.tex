%!TEX root = ../thesis.tex
%*******************************************************************************
%****************************** Second Chapter *********************************
%*******************************************************************************

\chapter{Lattice QCD}

\ifpdf
    \graphicspath{{Chapter2/Figs/Raster/}{Chapter2/Figs/PDF/}{Chapter2/Figs/}}
\else
    \graphicspath{{Chapter2/Figs/Vector/}{Chapter2/Figs/}}
\fi
Since the first efforts to construct a non-perturbative approach to QCD in 1974\cite{Wilson:1974sk}, lattice QCD has developed over the past 40 years into a powerful tool used to probe the low-energy behaviour of the strong nuclear force. Rather than treat spacetime as a set of continuous axes, it is instead discretised into a finite set of points on a four-dimensional hypercube. This prescription allows for the explicit calculation of path integrals present in QCD, at the cost of introducing finite-spacing errors that must be systematically accounted for. In this chapter we will discuss the behaviour of QCD when spacetime is continuous (hereafter referred to as the continuum), and demonstrate how the transition can be made to a finite set of coordinates on a lattice. We will also describe the two choices of gauge used in this research, and how they are applied to the lattice. 

\section{QCD in the Continuum}
QCD is the gauge field theory that describes the interactions of quarks and gluons. Like all gauge theories, it has an internal symmetry group which the Lagrangian is invariant under. In the case of QCD there are three quark colours, which leads to the symmetry group being $SU(3)$, the group of $3\times 3$ unitary matrices of determinant 1 \footnote{This description is only true in the fundamental representation of the group, but this is the symmetry we observe in the Lagrangian and is a useful way to visualise the group symmetry.}. We can see this $SU(3)$ symmetry by inspecting the QCD quark Lagrangian
%
\begin{equation}
\mathcal{L} = \bar{\psi}(x)\,(i\slashed{\partial}-m)\,\psi(x)\,.
\label{GlobalQuarkLagrangian}
\end{equation}
%
If we apply an $SU(3)$ transformation $\Omega$ to the colour indices of the quark ($\psi (x)$) and anti-quark ($\bar{\psi} (x)$) fields, we see that
%
\begin{align*}
\mathcal{L}\rightarrow\mathcal{L'}&=\bar{\psi}(x)\,\Omega^\dag\,(i\slashed{\partial}-m)\,\Omega\,\psi(x)\\
&= \bar{\psi}(x)\,(i\slashed{\partial}-m)\,\psi(x)\,\Omega^\dag\,\Omega\\
&= \bar{\psi}(x)\,(i\slashed{\partial}-m)\,\psi(x)\\
&= \mathcal{L}\, .
\end{align*}\\
%
If this symmetry were all we required then we would be done and our theory would be pleasantly simple. However, we find that we actually need our gauge symmetry to be \textit{local}; that is, we demand that our gauge transformation itself be a function of $x$\cite{peskin2018introduction}. In this case, we find that the derivative in Eq.~\ref{GlobalQuarkLagrangian} results in a loss of $SU(3)$ symmetry. To amend this, we introduce the notion of the gauge-covariant derivative
%
\begin{equation}
D_\mu = \partial_\mu - ig A_\mu(x)^a\,\frac{\lambda_a}{2}\, ,
\end{equation}
%
where the $A_\mu^a(x)$ are the eight new gauge fields and the $\lambda_a$ are the eight traceless, Hermitian, generators of the $SU(3)$ group, known as the Gell-Mann matrices. The generators of a group are related to the full group via the matrix exponential, such that any element $\Omega$ of $SU(3)$ can be written
\begin{equation}
\Omega = \exp\left(i\omega^a\,\frac{\lambda_a}{2}\right)
\end{equation}


\section{Pure Gauge Action}
For the purpose of this research, we are interested in the behaviour of gluons, and as such we need to develop a description of pure gauge fields. In the continuum, a pure gauge field has the Lagrangian\cite{ryder1996quantum}
%
\begin{equation}
\mathcal{L}=\frac{1}{2}F_{\mu\nu}\,F^{\mu\nu}
\label{QCDLagrangian}
\end{equation}
%
and corresponding action
\begin{equation}
\mathcal{S}=\int~d^4x~\mathcal{L},
\label{QCDAction}
\end{equation}
where $F_{\mu\nu}$ is the field-strength tensor, and can be written in terms of the traceless, Hermitian, gauge potential $A_\mu$ as
\begin{equation}
F_{\mu\nu}=\partial_\mu A_\nu - \partial_\nu A_\mu -ig[A_\mu,\,A_\nu]
\label{FieldStrengthTensor}
\end{equation}
When considering the path integral formulation of a gauge field theory, integrals such as the generating functional,
\begin{equation}
\mathcal{Z} =\int \mathcal{D} A_\mu \exp\left(i\,\mathcal{S}\,[A_\mu(x)]\right),
\end{equation}
and others of a similar form appear frequently. This integral closely resembles the partition function found in statistical mechanics, $Z_{\text{classical}}=\int d^3x\,d^3p\,\exp\left(-\beta\,H(x,p)\right)$, with the notable exception of the factor of $i$ in the exponential. This factor leads to an oscillatory weight term, rendering numerical simulations impossible. To ensure that the weight factor is purely real, it is necessary to perform a Wick rotation into Euclidean space\cite{Schafer:1996wv,Wilson:1974sk}
\begin{align*}
x_0\rightarrow -ix_0
\end{align*}
The generating functional now becomes 
\begin{equation}
\mathcal{Z}_{\text{Eucl}} =\int \mathcal{D} A_\mu \exp\left(-\mathcal{S}_{\text{Eucl}}\,[A_\mu(x)]\right).
\end{equation}
The Wick rotation also has the consequence of reducing the metric $g_{\mu\nu}$ to the identity, meaning that there is no longer any differentiation between covariant and contravariant tensors.\\

Within this framework, we can now consider discretising spacetime into a finite lattice, with each lattice site separated by a spacing $a$. When spacetime is discretised, it becomes necessary to consider derivatives as finite differences and integrals as finite sums. For example, we can construct the lattice form of Eq.~\ref{FieldStrengthTensor} as
\begin{equation}
F_{\text{Lat}}^{\mu\nu}(x) = \frac{A_\nu(x+a\hat{\mu})-A_\nu(x)}{a}-\frac{A_\mu(x+a\hat{\nu})-A_\mu(x)}{a}-ig[A_\mu(x),\,A_\nu(x)].
\label{DiscreteFST}
\end{equation}
The notation $A_\nu(x+a\hat{\mu})$ denotes the field $A_\nu$ located at the site one lattice spacing in the $\hat{\mu}$ direction from $x$.
\section{Gauge Fixing}
\subsection{Landau Gauge}
\subsection{Maximal Centre Gauge}


%\begin{landscape}
%
%\section*{Subplots}
%I can cite Wall-E (see Fig.~\ref{fig:WallE}) and Minions in despicable me (Fig.~\ref{fig:Minnion}) or I can cite the whole figure as Fig.~\ref{fig:animations}
%
%
%\begin{figure}
%  \centering
%  \begin{subfigure}[b]{0.3\textwidth}
%    \includegraphics[width=\textwidth]{TomandJerry}
%    \caption{Tom and Jerry}
%    \label{fig:TomJerry}   
%  \end{subfigure}             
%  \begin{subfigure}[b]{0.3\textwidth}
%    \includegraphics[width=\textwidth]{WallE}
%    \caption{Wall-E}
%    \label{fig:WallE}
%  \end{subfigure}             
%  \begin{subfigure}[b]{0.3\textwidth}
%    \includegraphics[width=\textwidth]{minion}
%    \caption{Minions}
%    \label{fig:Minnion}
%  \end{subfigure}
%  \caption{Best Animations}
%  \label{fig:animations}
%\end{figure}
%
%
%\end{landscape}

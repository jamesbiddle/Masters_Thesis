%!TEX root = ../thesis.tex
\chapter{Taylor expansion of $P_{\mu\nu}$}\label{app:TEPlaquette}
First, we recall the definitions of the gauge link in the continuum
%
\begin{equation}
U_\mu(x) = \mathcal{P}\exp\left(ig\int_x^{x+a\hat{\mu}}dx^\prime \, A_\mu(x^\prime)\right)  \, ,
\label{eq:GaugeLink(app)}
\end{equation}
%
and the plaquette formed from the product of the gauge links around a $1\times 1$ loop
\begin{equation}
P_{\mu\nu} = U_\mu(x)\,U_\nu(x+a\hat{\mu})\,U^\dagger_\mu(x+a\hat{\nu})\,U^\dagger_\nu(x)\, .
\label{eq:Plaquette(app)}
\end{equation} 

We can approximate the integral in Eq.~\ref{eq:GaugeLink(app)} by Taylor expanding $A_\mu(x^\prime)$ around $x^\prime=x$ and explicitly evaluating the integral. Note that once we have Taylor expanded $A_\mu(x^\prime)$, the term within the integral becomes Abelian for all values of $x^\prime$, allowing us to omit the path ordering from Eq.~\ref{eq:GaugeLink(app)}. Performing the expansion, we find that
%
\begin{align}
U_\mu(x)&=\exp\left(ig\int_x^{x+a\hat{\mu}} dx^\prime A_\mu\left(x\right) + \left(\partial^\prime_\sigma A_\mu(x^\prime)\right)\big|_{x^\prime = x}\,(x^{\prime\sigma} - x^\sigma) + \mathcal{O}(a^3)\right)\nonumber\\
&=\exp\left(iag A_\mu\left(x\right) + \frac{1}{2}ia^2 g\, \partial_\mu A_\mu\left(x\right)\right)\, . \label{eq:UTaylor}
\end{align}
%
Note that $\sigma$ index is summed over, however only the $x^\mu$ term survives the integration, as for all $x^\sigma$, $\sigma\neq \mu$, $(x^{\prime\sigma}-x^\sigma)=0$. Similarly, we evaluate
%
\begin{align}
U_\nu(x+a\hat{\mu}) &= \exp\left(ig\int_{x+a\hat{\mu}}^{x+a\hat{\mu}+a\hat{\nu}} dx^\prime A_\nu\left(x\right) + \left(\partial^\prime_\sigma A_\nu(x^\prime)\right)\big|_{x^\prime = x}\,(x^{\prime\sigma} - x^\sigma) + \mathcal{O}(a^3)\right)\nonumber\\
&= \exp\left( iag A_\nu(x) + iag\,\partial_\mu A_\nu (x) + ig\int_{x_\nu}^{x_\nu+a} dx^\prime_\nu \, \partial_\nu A_\nu (x) (x^{\prime\nu} - x^\nu)\right)\nonumber\\
&= \exp\left(iag A_\nu(x) + ia^2g\,\partial_\mu A_\nu(x) + \frac{1}{2}ia^2 g\,\partial_\nu A_\nu(x)\right)\, ,\label{eq:UTaylor2}
\end{align}
%
where once again the $\sigma$ index is summed over. We retain the $\partial_\mu A_\nu (x)$ term as for $\sigma = \mu$ we have $(x^{\prime\mu} - x^\mu) = (x^\mu + a\hat{\mu} - x^\mu) = a\hat{\mu}$.\\

We will also require the Baker-Campbell-Hausdorff identity for non-Abelian matrix exponentials
%
\begin{equation}
\exp(A)\,\exp(B) = \exp\left(A + B +\frac{1}{2}[A,\,B]\right)\, .
\end{equation}
%
Substituting Eq.~\ref{eq:UTaylor} and Eq.~\ref{eq:UTaylor2} into Eq.~\ref{eq:Plaquette(app)} and retaining only terms up to $\mathcal{O}(a^2)$ we have
%
\begin{alignat*}{2}
P_{\mu\nu} &= &&\exp\left(ig\left(a\,A_\mu(x)+\frac{1}{2}a^2\,\partial_\mu A_\mu(x) \right)\right)\\
& &&\times\exp\left(ig\left(aA_\nu(x) + \frac{1}{2}a^2\,\partial_\nu A_\nu(x) + a^2\,\partial_\mu A_\nu(x)\right)\right)\\
& &&\times\exp\left(-ig\left(aA_\mu(x) + \frac{1}{2}a^2\,\partial_\mu A_\mu(x) + a^2\,\partial_\nu A_\mu(x)\right)\right)\\
& &&\times\exp\left(-ig\left(a\,A_\nu(x)+\frac{1}{2}a^2\,\partial_\mu A_\nu(x) + \right)\right)\\
&= &&\exp\left(ia^2g\,\partial_\mu A_\nu(x) - ia^2g\,\partial_\nu A_\mu(x) -\frac{a^2g^2}{2}[A_\mu(x),\,A_\nu(x)] + \frac{a^2g^2}{2}[A_\nu(x),\,A_\mu(x)]\right)\\
&= &&\exp\left(ia^2g F_{\mu\nu}(x)\right)\\
&\approx &&\, I + ia^2 g F_{\mu\nu} -a^4 g^2 F_{\mu\nu}^2\, .
\end{alignat*}

\chapter{Evaluation of $\operatorname{Re}\Tr(F_1(U)^\dagger \,U)$}\label{app:CoolingMaximise}

It is apparent that we can write Eq.~\ref{eq:MaximisedCooling} in the form
\begin{equation}
\Re\Tr(F_1(U)^\dagger\, U)\, ,
\end{equation}
where $U = U_\mu\, \bar{U}$. Expanding this, we have
\begin{align*}
\Re\Tr(F_1(U)^\dagger\, U) &=\Re \left( \frac{1}{2}\,U_{11} \left(U_{22} + U_{11}^*\right) - \frac{1}{2}\,U_{21} \left(U_{12} - U_{21}^*\right)\right.\\
&~~~~+ \left. \frac{1}{2}\,U_{22} \left(U_{11} + U_{22}^*\right)
- \frac{1}{2}\,U_{12} \left(U_{21} - U_{12}^*\right) +U_{33}\right)\\
&= \Re\left(\frac{|U_{11}|^2}{2} + \frac{|U_{22}|^2}{2} + \frac{|U_{12}|^2}{2} + \frac{|U_{21}|^2}{2} + U_{11}\,U_{22} - U_{12}\,U_{21} + U_{33}\right)\, .
\end{align*}
We now wish to make use of the known determinant of $F_1(U)$ to simplify this expression.
\begin{align*}
\det(F_1(U)) &= \frac{1}{4} \left(U_{11}+ U_{22}^*\right)\left(U_{11}^* + U_{22}\right)\\
&~~~~-\frac{1}{4}\left(U_{12} - U_{21}^*\right)\left( U_{21} - U_{12}^*\right)\\
&=\frac{1}{4}\left( |U_{11}|^2 + |U_{22}|^2 + |U_{12}|^2 + |U_{21}|^2\right.\\
&~~~~+ U_{11}\,U_{22} + U_{11}^*\,U_{22}^* + U_{12}\,U_{21} + U_{12}^*\,U_{21}^*\big) = 1\, ,
\end{align*}
and hence
\begin{equation}
\Re(\det(F_1(U))) = \frac{1}{2}\Re\left(\frac{|U_{11}|^2}{2} + \frac{|U_{22}|^2}{2} + \frac{|U_{12}|^2}{2} + \frac{|U_{21}|^2}{2} + U_{11}\,U_{22} - U_{12}\,U_{21}\right) = 1\, .
\end{equation}
So we find the desired result
\begin{equation}
\Re\Tr(F_1(U)^\dagger\, U) = 2 + \Re(U_{33})
\end{equation}
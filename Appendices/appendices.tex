%!TEX root = ../thesis.tex
\chapter{Supplementary Material}
\section{Matrix definitions}\label{app:GellMann}
The standard form of the $2\times 2$ Pauli matrices is
%
\begin{equation}
\sigma_1 = \begin{pmatrix}
0 & 1\\
1 & 0
\end{pmatrix},
\hspace{0.5cm}
\sigma_2 = \begin{pmatrix}
0 & -i\\
i & 0
\end{pmatrix},
\hspace{0.5cm}
\sigma_3 = \begin{pmatrix}
1 & 0\\
0 & -1
\end{pmatrix}\\, .
\end{equation}
%
The $4\times 4$ gamma matrices in the Dirac representation in Euclidean space are therefore given by
%
\begin{equation}
\gamma_i = \begin{pmatrix}
0 & \sigma_i\\
-\sigma_i & 0
\end{pmatrix},
\hspace{0.5cm}
\gamma_4 = \begin{pmatrix}
I & 0\\
0 & -I
\end{pmatrix}\, .
\end{equation} 
%
The gamma matrices in this representation are evidently Hermitian, and satisfy the anti-commutation relationship,
%
\begin{equation}
\lbrace \gamma_\mu\, ,\, \gamma_\mu \rbrace = 2\,\delta_{\mu\nu}\, .
\end{equation}
%
The Gell-Mann matrices $\lambda_a$ are related to the generators of $SU(3)$, $t_a$, by~\cite{peskin2018introduction}
%
\begin{equation}
t_a = \frac{\lambda_a}{2}\, .
\end{equation}
%
The matrices are given by
%
\begin{align}
\lambda _ { 1 } &= \left( \begin{array} { c c c } { 0 } & { 1 } & { 0 } \\ { 1 } & { 0 } & { 0 } \\ { 0 } & { 0 } & { 0 } \end{array} \right) \quad \lambda _ { 2 } = \left( \begin{array} { c c c } { 0 } & { - i } & { 0 } \\ { i } & { 0 } & { 0 } \\ { 0 } & { 0 } & { 0 } \end{array} \right) \quad \lambda _ { 3 } = \left( \begin{array} { c c c } { 1 } & { 0 } & { 0 } \\ { 0 } & { - 1 } & { 0 } \\ { 0 } & { 0 } & { 0 } \end{array} \right)\nonumber\\
\lambda _ { 4 } &= \left( \begin{array} { c c c } { 0 } & { 0 } & { 1 } \\ { 0 } & { 0 } & { 0 } \\ { 1 } & { 0 } & { 0 } \end{array} \right) \quad \lambda _ { 5 } = \left( \begin{array} { c c c } { 0 } & { 0 } & { - i } \\ { 0 } & { 0 } & { 0 } \\ { i } & { 0 } & { 0 } \end{array} \right)\nonumber\\
\lambda _ { 6 } &= \left( \begin{array} { c c c } { 0 } & { 0 } & { 0 } \\ { 0 } & { 0 } & { 1 } \\ { 0 } & { 1 } & { 0 } \end{array} \right) \quad \lambda _ { 7 } = \left( \begin{array} { c c c } { 0 } & { 0 } & { 0 } \\ { 0 } & { 0 } & { - i } \\ { 0 } & { i } & { 0 } \end{array} \right) \quad \lambda _ { 8 } = \frac { 1 } { \sqrt { 3 } } \left( \begin{array} { c c c } { 1 } & { 0 } & { 0 } \\ { 0 } & { 1 } & { 0 } \\ { 0 } & { 0 } & { - 2 } \end{array} \right)\, .
\end{align}

\section{Gauge transformation of $F_{\mu\nu}$}\label{app:GTF}
Recall from Eq.~\eqref{eq:FieldStrengthTensor} that the field strength tensor is defined to be
%
\begin{equation}
F_{\mu\nu}=\partial_\mu A_\nu - \partial_\nu A_\mu + ig[A_\mu,\,A_\nu]\, .
\label{eq:FieldStrength(app)}
\end{equation}
%
To calculate how $F_{\mu\nu}$ transforms under a gauge transformation, we will also make use of the gauge transformation property for $A_\mu$,
%
\begin{equation}
A_\mu \rightarrow \Omega\,A_\mu\,\Omega^\dag + \frac{i}{g}\,(\partial_\mu\,\Omega)\,\Omega^\dag\, . 
\label{eq:PotentialGT(app)}
\end{equation}
%
We will also make repeated use of the unitarity of $\Omega$, specifically the fact that
%
\begin{align}
\partial_\mu \left(\Omega\, \Omega^\dagger\right) &= \left(\partial_\mu\,\Omega\right)\Omega^\dagger + \Omega\left(\partial_\mu\, \Omega^\dagger\right)\nonumber\\
&=\partial_\mu \, I\nonumber\\
&= 0\nonumber\\
\implies & \left(\partial_\mu\,\Omega\right)\Omega^\dagger = -\Omega\left(\partial_\mu\, \Omega^\dagger\right)\, .
\end{align}
%
Substituting Eq.~\eqref{eq:PotentialGT(app)} into Eq.~\eqref{eq:FieldStrength(app)} we obtain
%
\begin{align}
F_{\mu\nu} \rightarrow &\left(\partial_\mu \, \Omega\right)A_\nu \, \Omega^\dagger + \Omega\left(\partial_\mu\,A_\nu\right)\Omega^\dagger + \Omega \, A_\nu\left(\partial_\mu \, \Omega^\dagger\right)\nonumber\\
& + \frac{i}{g}\left(\partial_\nu\,\partial_\mu\, \Omega\right)\Omega^\dagger + \frac{i}{g}\left(\partial_\nu\,\Omega\right)\left(\partial_\mu\,\Omega^\dagger\right)\nonumber\\
& - \left(\partial_\nu \, \Omega\right)A_\mu \, \Omega^\dagger - \Omega\left(\partial_\nu\,A_\mu\right)\Omega^\dagger - \Omega \, A_\mu\left(\partial_\nu \, \Omega^\dagger\right)\nonumber\\
& - \frac{i}{g}\left(\partial_\nu\,\partial_\mu\, \Omega\right)\Omega^\dagger - \frac{i}{g}\left(\partial_\mu\,\Omega\right)\left(\partial_\nu\,\Omega^\dagger\right)\nonumber\\
& +ig \, \Omega \, A_\mu \, A_\nu \, \Omega^\dagger - \left(\partial_\mu \, \Omega\right)A_\nu \, \Omega^\dagger + \Omega\,A_\mu\left(\partial_\nu \, \Omega^\dagger\right) + \frac{i}{g}\left(\partial_\mu\,\Omega\right)\left(\partial_\nu\,\Omega^\dagger\right)\nonumber\\
& -ig \, \Omega \, A_\nu \, A_\mu \, \Omega^\dagger + \left(\partial_\nu \, \Omega\right)A_\mu \, \Omega^\dagger - \Omega\,A_\nu\left(\partial_\mu \, \Omega^\dagger\right) - \frac{i}{g}\left(\partial_\nu\,\Omega\right)\left(\partial_\mu\,\Omega^\dagger\right)\, . 
\end{align}
Cancelling off terms reduces the above expression to the desired result,
%
\begin{align}
F_{\mu\nu} \rightarrow &\, \Omega\left(\partial_\mu\,A_\nu\right)\Omega^\dagger - \Omega\left(\partial_\nu\,A_\mu\right)\Omega^\dagger +ig\,\Omega\left[A_\mu,\,A_\nu\right]\Omega^\dagger\nonumber\\
&= \Omega\,F_{\mu\nu}\,\Omega^\dagger\, .
\end{align}

\section{Wilson line gauge transformation}\label{app:WilsonLineGT}

We wish to show that the Wilson line obeys the gauge transformation property
%
\begin{equation}
U_\mu(x) \rightarrow \Omega(x)\,U_\mu(x)\,\Omega^\dagger(x)
\end{equation}
%
We can apply a gauge transformation to $A_\mu$ to obtain
%
\begin{equation}
U_\mu(x)\rightarrow\mathcal{P}\exp\left(-iag\int_0^1\,dt\, \Omega(x(t))\,A_\mu(x(t))\,\Omega^\dagger(x(t)) + \frac{i}{g}\,(\partial_\mu\,\Omega(x(t)))\,\Omega^\dagger(x(t))\right)\, ,
\label{eq:WilsonLineGT(app)}
\end{equation}
%
where $x(t)=x+at\hat{\mu}$. To simplify this expression, we need to make use of an equivalent expression for the path-ordered exponential. For a generic path-ordered exponential of a function $a(t)$ we can write
%
\begin{align}
&\mathcal{P}\left(\int_0^t dt^\prime\, a(t^\prime)\right) = \lim_{N\rightarrow\infty} \left( e^{a(t_N)\Delta t}\,e^{a(t_{N-1})\Delta t}\,\cdots\,e^{a(t_0)\Delta t}\right)\nonumber\\
\implies &\mathcal{P}\left(-\int_0^t dt^\prime\, a(t^\prime)\right) = \lim_{N\rightarrow\infty} \left( e^{a(t_0)\Delta t}\,e^{a(t_{1})\Delta t}\,\cdots\,e^{a(t_N)\Delta t}\right)\, , \label{eq:OrderedExponentialLimit}
\end{align}
%
where $\lbrace t_0 = 0, \, \cdots\, , \, t_N = t\rbrace$ is a partition of the integration range into equal slices of length $\Delta t = \frac{t}{N}$. We also wish to utilise the fact that
%
\begin{equation}
e^{ig\,\Omega(x)\,A_\mu(x)\,\Omega^\dagger(x)} = \Omega(x)\,e^{ig\,A_\mu(x)}\,\Omega^\dagger(x)\, ,
\end{equation}
as can be easily derived from writing the exponential as an infinite sum.\\

Writing Eq.~\eqref{eq:WilsonLineGT(app)} in the form of Eq.~\eqref{eq:OrderedExponentialLimit} and employing the fact that path-ordering permits us to only retain the first order terms in the Baker-Campbell-Hausdorff identity (see Eq.~\eqref{eq:BCH}), we find that
%
\begin{align}
U_\mu(x)\rightarrow \lim_{N\rightarrow\infty} &\left( \Omega(x_0)\,e^{ig\,A_\mu(x_0)\,\Delta x}\,\Omega^\dagger(x_0) \exp\left((\partial_\mu\,\Omega(x_0))\,\Omega^\dagger(x_0)\right)\right.\nonumber\\
&\times \Omega(x_1)\,e^{ig\,A_\mu(x_1)\,\Delta x}\,\Omega^\dagger(x_1) \exp\left((\partial_\mu\,\Omega(x_1))\,\Omega^\dagger(x_1)\Delta x\right) \times \, \cdots \nonumber\\
&\left. \times \, \Omega(x_{N-1})\,e^{ig\,A_\mu(x_{N-1})\,\Delta x}\,\Omega^\dagger(x_{N-1}) \exp\left((\partial_\mu\,\Omega(x_{N-1}))\,\Omega^\dagger(x_{N-1})\Delta x\right) \right)\, , \label{eq:WilsonLineExpansion(app)}
\end{align}
%
where $\Delta x = \frac{a}{N}$, $x_0=x$ and $x_N = x+a\hat{\mu}$. However, in the limit as $N\rightarrow\infty$, $\exp\left((\partial_\mu\,\Omega(x_i))\,\Omega^\dagger(x_i)\Delta x \right)$ is precisely the parallel transport operator for $\Omega$ over the distance $\Delta x$, and hence satisfies 
%
\begin{equation}
\Omega^\dagger(x_i)\exp\left((\partial_\mu\,\Omega(x_i))\,\Omega^\dagger(x_i)\Delta x \right) = \Omega^\dagger(x_{i+1})\, .
\label{eq:RotationTransport(app)}
\end{equation}
%
Substituting Eq.~\eqref{eq:RotationTransport(app)} into Eq.~\eqref{eq:WilsonLineExpansion(app)} eliminates all the gauge transformation terms except for the first and last transformations. This then reduces to our desired result,
%
\begin{equation}
U_\mu(x)\rightarrow \Omega(x)\,U_\mu(x)\,\Omega^\dagger(x+a\hat{\mu})\, .
\end{equation} 

\section{Taylor expansion of $P_{\mu\nu}$}\label{app:TEPlaquette}
Here we sketch out how to construct the field strength tensor in terms of the plaquette. We will neglect a discussion of the $\mathcal{O}(a^3)$ terms found in Eq.~\eqref{eq:PlaquetteExponantial(app)} as a more careful treatment outside the scope of this work is required to show that these higher order terms do not render the expansion shown in Eq.~\eqref{eq:PlaquetteExpansion} incorrect. However, this derivation highlights the key steps in arriving at the desired result. First, we recall the definitions of the gauge link in the continuum
%
\begin{equation}
U_\mu(x) = \mathcal{P}\exp\left(-iag\int_0^1 \,dt \,A_\mu(x + at\hat{\mu})\right)  \, ,
\label{eq:GaugeLink(app)}
\end{equation}
%
and the plaquette formed from the product of the gauge links around a $1\times 1$ loop
\begin{equation}
P_{\mu\nu} = U_\mu(x)\,U_\nu(x+a\hat{\mu})\,U^\dagger_\mu(x+a\hat{\nu})\,U^\dagger_\nu(x)\, .
\label{eq:Plaquette(app)}
\end{equation}
%
We can approximate the integral in Eq.~\eqref{eq:GaugeLink(app)} by Taylor expanding $A_\mu$ around $a=0$ and explicitly evaluating the integral. Note that once we have Taylor expanded $A_\mu$, the term within the integral commutes with itself for all values of $t$, allowing us to omit the path ordering from Eq.~\eqref{eq:GaugeLink(app)}. Performing the expansion, we find that
%
\begin{align}
U_\mu(x)&=\exp\left(-iag\int_0^1 dt \left(A_\mu\left(x\right) + at\partial_\mu A_\mu(x) + \mathcal{O}(a^3)\right)\right)\nonumber\\
&=\exp\left(-iag A_\mu\left(x\right) - \frac{1}{2}ia^2 g\, \partial_\mu A_\mu\left(x\right)+ \mathcal{O}(a^3)\right)\, . \label{eq:UTaylor}
\end{align}
%
Similarly, we evaluate
%
\begin{align}
U_\nu(x+a\hat{\mu}) &= \exp\left(-iag\int_0^1 dt \left(A_\nu(x) + a\partial_\mu A_\nu + at\partial_\nu A_\nu + \mathcal{O}(a^3)\right)\right)\nonumber\\
&= \exp\left(-iag A_\nu(x) - ia^2g\,\partial_\mu A_\nu(x) - \frac{1}{2}ia^2 g\,\partial_\nu A_\nu(x)+ \mathcal{O}(a^3)\right)\, . \label{eq:UTaylor2}
\end{align}
%
We will also require the Baker-Campbell-Hausdorff identity for non-Abelian matrix exponentials
%
\begin{equation}
\exp(A)\,\exp(B) = \exp\left(A + B +\frac{1}{2}[A,\,B]\right)\, .
\label{eq:BCH}
\end{equation}
%
Substituting Eq.~\eqref{eq:UTaylor} and Eq.~\eqref{eq:UTaylor2} into Eq.~\eqref{eq:Plaquette(app)} and retaining only terms up to $\mathcal{O}(a^2)$ we find that
%
\begin{alignat}{2}
P_{\mu\nu} &\simeq &&\exp\left(-ig\left(a\,A_\mu(x)+\frac{1}{2}a^2\,\partial_\mu A_\mu(x) \right)\right)\nonumber\\
& &&\times\exp\left(-ig\left(aA_\nu(x) + \frac{1}{2}a^2\,\partial_\nu A_\nu(x) + a^2\,\partial_\mu A_\nu(x)\right)\right)\nonumber\\
& &&\times\exp\left(ig\left(aA_\mu(x) + \frac{1}{2}a^2\,\partial_\mu A_\mu(x) + a^2\,\partial_\nu A_\mu(x)\right)\right)\nonumber\\
& &&\times\exp\left(ig\left(a\,A_\nu(x)+\frac{1}{2}a^2\,\partial_\nu A_\nu(x)\right)\right)\nonumber\\
&\simeq &&\exp\left(-ia^2g\,\partial_\mu A_\nu(x) + ia^2g\,\partial_\nu A_\mu(x) +\frac{a^2g^2}{2}[A_\mu(x),\,A_\nu(x)] - \frac{a^2g^2}{2}[A_\nu(x),\,A_\mu(x)]\right)\nonumber\\
&= &&\exp\left(-ia^2g F_{\mu\nu}(x) + \mathcal{O}(a^3)\right)\, .
\label{eq:PlaquetteExponantial(app)}
\end{alignat}
\section{Properties of the adjoint representation}
\label{app:RepMapProof}
Consider the mapping $H:SU(3)^\text{fundamental}\rightarrow SU(3)^\text{adjoint}$ defined by
%
\begin{equation}
\left[H(U)\right]_{ij} = \frac{1}{2}\Tr\left(\lambda_i \, U \, \lambda_j\, U^\dagger\right)\, .
\label{eq:FundMap(app)}
\end{equation}
%
We want to show that for $U,V\in SU(3)^F$
%
\begin{equation}
\left[H(U)\right]_{ij}\left[H(V)\right]^{jk} = \left[H(U V)\right]_i^{~k}\, .
\label{eq:MapPreserve(app)}
\end{equation}
%
To do this, we will need to make use of the following Fierz completeness relations for the $SU(3)$  generators
%
\begin{equation}
\lambda _ { b } ^ { a } \lambda _ { c } ^ { d } = 2\, \delta _ { c } ^ { a } \delta _ { b } ^ { d } - \frac { 2 } { 3 } \delta _ { b } ^ { a } \delta _ { c } ^ { d }\, .
\end{equation}
%
Substituting Eq.~\eqref{eq:FundMap(app)} into Eq.~\eqref{eq:MapPreserve(app)} and noting that repeated indicies are summed over, we have
\begin{align}
\left[H(U)\right]_{\alpha\beta}\left[H(V)\right]^{\beta\gamma} &= \frac{1}{2} \lambda^\alpha_{ab}\, U_{bc}\,\lambda^\beta_{cd}\, U^\dagger_{da} \times \frac{1}{2} \lambda^\beta_{ef}\, V_{fg}\,\lambda^\gamma_{gh}\, V^\dagger_{he}\nonumber\\
&=\frac{1}{2} U_{bc}\,U^\dagger_{da}\,V_{fg}\,V^\dagger_{he}\,\lambda^\alpha_{ab}\, \lambda^\gamma_{gh} \left(\delta_{cf}\,\delta_{de} - \frac{1}{3}\delta_{cd}\delta_{ef}\right)\nonumber\\
&=\frac{1}{2}U_{bc} \, V_{cg} \, \lambda^\gamma_{gh} \, V^\dagger_{hd} \, U^\dagger_{da} \, \lambda^\alpha_{ab} - \frac{1}{6} U_{bc} \, U^\dagger_{ca} \, V^\dagger_{he} \, V_{eg} \lambda^\alpha_{ab}\, \lambda^\gamma_{gh}\nonumber\\
&= \frac{1}{2}U_{bc} \, V_{cg} \, \lambda^\gamma_{gh} \, V^\dagger_{hd} \, U^\dagger_{da} \, \lambda^\alpha_{ab} - \frac{1}{6}\delta_{ba}\delta_{hg} \lambda^\alpha_{ab}\, \lambda^\gamma_{gh}\nonumber\\
&=\frac{1}{2}\Tr\left(U \, V \, \lambda^\gamma \, (U \, V)^\dagger \lambda^\alpha\right) - \frac{1}{6}\Tr\left(\lambda^\alpha\right) \, \Tr\left(\lambda^\gamma\right)\, .
\end{align}
Making use of the cyclic property of the trace and the fact that the Gell-Mann matrices are traceless, we find the desired result,
%
\begin{align}
\left[H(U)\right]_{\alpha\beta}\left[H(V)\right]^{\beta\gamma} &= \frac{1}{2}\Tr\left(\lambda^\alpha U \, V \, \lambda^\gamma \, (U \, V)^\dagger \right)\nonumber\\
&= \left[H(UV)\right]_{\alpha\gamma} \, .
\end{align}\\

We also wish to show that for $U^A\in SU(3)^A$ and $U\in SU(3)^F$ that
%
\begin{equation}
\Tr\left(U^A\right) = \left|\Tr(U)\right|^2 -1\, .
\end{equation}
%
Making use of Eq.~\eqref{eq:FundMap(app)}, we have
\begin{align}
\Tr\left(U^A\right) &= \sum_{\alpha=1}^8 \frac{1}{2}\Tr\left(\lambda^\alpha \, U_\mu(x) \, \lambda^\alpha \, U_\mu^\dagger(x)\right)\nonumber\\
&= \frac{1}{2}U_{bc} \, U^\dagger_{da} \, \lambda_{ab}^\alpha \, \lambda_{cd}^\alpha\nonumber\\
&= U_{bc} \, U^\dagger_{da} \left(\delta_{ad}\,\delta_{bc} - \frac{1}{3}\delta_{ab}\,\delta_{cd}\right)\nonumber\\
&= \Tr\left(U\right) \, \Tr\left( U^\dagger \right) - \frac{1}{3}\Tr\left(U\, U^\dagger\right)\nonumber\\
&= \left|\Tr(U)\right|^2 -1
\end{align}

\section{Cooling algorithm derivation}\label{app:Cooling}
We wish to find $a_i$ such that 
%
\begin{equation}
U_\mu^\prime = a_3 \, a_2 \, a_1 \, U_\mu\, ,
\label{eq:UPrime(app)}
\end{equation}
%
minimises the local Wilson action associated with the link $U_\mu$. This minimisation is equivalent to maximising
%
\begin{equation}
R = \operatorname{Re} \Tr (U_\mu^\prime \, \bar{U})\, .
\label{eq:CoolFunctional(app)}
\end{equation}
%
To determine the optimal choice for the $a_i$, we define the following three functions, $F_i(V)$, such that $F_i : SU(3)\rightarrow SU(2)$:
%
\begin{align}
F_1(V) &=\frac{1}{k_1} \begin{pmatrix}
\frac{1}{2}\left(V_{11} + V_{22}^*\right) & \frac{1}{2}\left(V_{12} - V_{21}^*\right) & 0\\
\frac{1}{2}\left(V_{21} - V_{12}^*\right) & \frac{1}{2}\left(V_{11}^* + V_{22}\right) & 0\\
0 & 0 & k_1
\end{pmatrix}\, ,\\
F_2(V) &=\frac{1}{k_2} \begin{pmatrix}
\frac{1}{2}\left(V_{11} + V_{33}^*\right) & 0 &\frac{1}{2}\left(V_{13} - V_{31}^*\right)\\
0 & k_2 & 0\\
\frac{1}{2}\left(V_{31} - V_{13}^*\right) & 0 & \frac{1}{2}\left(V_{11}^* + V_{33}\right)
\end{pmatrix}\, ,\\
F_3(V) &=\frac{1}{k_3} \begin{pmatrix}
k_3 & 0 & 0\\
0 & \frac{1}{2}\left(V_{22} + V_{33}^*\right) & \frac{1}{2}\left(V_{23} - V_{32}^*\right)\\
0 & \frac{1}{2}\left(V_{32} - V_{23}^*\right) & \frac{1}{2}\left(V_{22}^* + V_{33}\right)
\end{pmatrix}\, ,
\end{align}
%
where $k_i^2$ is the determinant of the $2\times 2$ $SU(2)$ sub-block. The $\frac{1}{k_i}$ factor therefore fixes the determinant such that $\det(F_i(V))=1$. We now wish to find a suitable $V$ to define our $a_i$'s. Consider the first case, $U^{(1)}_\mu = a_1\,U_\mu$, and let $a_1 = F_1(U_\mu\,\bar{U})^\dagger$. It is worth stating explicitly that it is this step where the fact that a sum of $SU(2)$ matrices is proportional to an $SU(2)$ matrix is utilised. Despite the fact that $U_\mu\,\bar{U} \notin SU(3)$, $U_\mu \, \tilde{U}_\alpha\in SU(3)~\forall\, \alpha$, which implies that $F_i(U_\mu\, \tilde{U}_\alpha)\in SU(2)$. Then we have, for example,
%
\begin{align}
F_1(U_\mu \, \bar{U}) &= \frac{1}{k_1}\sum_\alpha
\begin{pmatrix}
\frac{1}{2}\left((U_\mu \, \tilde{U}_\alpha)_{11} + (U_\mu \, \tilde{U}_\alpha)^*_{22}\right) & \frac{1}{2}\left((U_\mu \, \tilde{U}_\alpha)_{12} - (U_\mu \, \tilde{U}_\alpha)^*_{21}\right) & 0\\
\frac{1}{2}\left((U_\mu \, \tilde{U}_\alpha)_{21} - (U_\mu \, \tilde{U}_\alpha)^*_{12}\right) & \frac{1}{2}\left((U_\mu \, \tilde{U}_\alpha)^*_{11} + (U_\mu \, \tilde{U}_\alpha)_{22}\right) & 0\\
0 & 0 & k_1
\end{pmatrix}\nonumber\\
&=\frac{1}{k_1}\sum_\alpha
\begin{pmatrix}
k_{1,\,\alpha}\,\left(F_1(U_\mu\,\tilde{U}_\alpha)\right)_{11} & k_{1,\,\alpha}\,\left(F_1(U_\mu\,\tilde{U}_\alpha)\right)_{12} & 0\\
k_{1,\,\alpha}\,\left(F_1(U_\mu\,\tilde{U}_\alpha)\right)_{21} & k_{1,\,\alpha}\,\left(F_1(U_\mu\,\tilde{U}_\alpha)\right)_{22} & 0\\
0 & 0 & k_1
\end{pmatrix}\nonumber\\
&=\frac{1}{k_1}\sum_\alpha
\left(\begin{array}{@{}c|c@{}}
k_{1,\,\alpha} \,F_1(U_\mu\,\tilde{U}_\alpha)_{2\times 2} & 0\\
  \hline 
0 & k_1
\end{array}\right)
\label{eq:FExpansion}
\end{align}
%
The $2\times 2$ block in Eq.~\eqref{eq:FExpansion} is a sum over terms proportional to $SU(2)$ matrices, and hence the result is proportional to an $SU(2)$ matrix. Thus with the appropriate normalisation from the $k_1$ factor, we see that $F_1(U_\mu \, \bar{U})\in SU(2)$. The same result holds true for $F_2$ and $F_3$.\\

With the definition $a_1 = F_1(U_\mu\,\bar{U})^\dagger$, the functional in Eq.~\eqref{eq:CoolFunctional(app)} we are seeking to maximise can now be directly evaluated. Setting $U = U_\mu\,\bar{U}$, we can write Eq.~\eqref{eq:CoolFunctional(app)} as 
%
\begin{align}
\Re\Tr(F_1(U)^\dagger\, U) &=\Re \left(\frac{1}{k_1}\left[ \frac{1}{2}\,U_{11} \left(U_{22} + U_{11}^*\right) - \frac{1}{2}\,U_{21} \left(U_{12} - U_{21}^*\right)\right.\right.\nonumber\\
&~~~~+ \left. \left.\frac{1}{2}\,U_{22} \left(U_{11} + U_{22}^*\right)
- \frac{1}{2}\,U_{12} \left(U_{21} - U_{12}^*\right) +k_1\,U_{33}\right]\right)\nonumber\\
&= \frac{1}{k_1}\Re\left(\frac{|U_{11}|^2}{2} + \frac{|U_{22}|^2}{2} + \frac{|U_{12}|^2}{2} + \frac{|U_{21}|^2}{2} + U_{11}\,U_{22} - U_{12}\,U_{21} + k_1\,U_{33}\right)\, .\label{eq:FunctionalExpanded}
\end{align}
%
Here we have used the fact that by the construction of $F_1(U)$, $k_1$ is real and can therefore be brought to the front of Eq.~\eqref{eq:FunctionalExpanded}. We now wish to make use of the known determinant of $F_1(U)$ to simplify this expression.
%
\begin{align}
\det(F_1(U)) &= \frac{1}{4\,k_1^2} \left(\left(U_{11}+ U_{22}^*\right)\left(U_{11}^* + U_{22}\right)-\left(U_{12} - U_{21}^*\right)\left( U_{21} - U_{12}^*\right)\right)\nonumber\\
&=\frac{1}{4\,k_1^2}\left( |U_{11}|^2 + |U_{22}|^2 + |U_{12}|^2 + |U_{21}|^2\right.\\
&~~~~+ U_{11}\,U_{22} + U_{11}^*\,U_{22}^* + U_{12}\,U_{21} + U_{12}^*\,U_{21}^*\big)\nonumber\\
&= \frac{1}{2\,k_1^2}\Re\left(\frac{|U_{11}|^2}{2} + \frac{|U_{22}|^2}{2} + \frac{|U_{12}|^2}{2} + \frac{|U_{21}|^2}{2} + U_{11}\,U_{22} - U_{12}\,U_{21}\right)\nonumber\\
&=1\, .
\end{align}
%
Substituting this determinant back into Eq.~\eqref{eq:FunctionalExpanded}, we find that
%
\begin{align}
\Re\Tr(F_1(U)^\dagger\, U) &= \Re\left(\frac{1}{k_1}(2k_1^2 + k_1\,U_{33}))\right)\nonumber\\
&=2\,k_1 + \Re\left(U_{33}\right)\, .
\end{align}
%
Finally, substituting back $U = U_\mu\,\bar{U}$ brings us to the desired result,
%
\begin{equation}
\Re \Tr (F_1(U_\mu\,\bar{U})^\dagger \, U_\mu \, \bar{U}) = 2k_1 + \Re\left( (U_\mu\, \bar{U})_{33}\right)\, .
\label{eq:MaximisedCooling}
\end{equation}
%
By the matrix structure of $F_1(V)$ it is apparent that $\Re(U_\mu\, \bar{U})_{33}$ is invariant under pre-multiplication by $a_1$, so it is clear that Eq.~\eqref{eq:MaximisedCooling} represents the maximum attainable value for this form of $U^{(1)}_\mu$. Similarly, we let $a_2=F_2(U_\mu\,\bar{U})^\dagger$ and $a_3=F_3(U_\mu\,\bar{U})^\dagger$ to obtain the final value of $U^\prime_\mu$ according to Eq.~\eqref{eq:UPrime(app)}. This construction results in a choice of $U_\mu^\prime \in SU(3)$ that minimises the local Wilson action.
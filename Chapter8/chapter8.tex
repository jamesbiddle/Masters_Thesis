%!TEX root = ../thesis.tex
%*******************************************************************************
%*********************************** Eighth Chapter *****************************
%*******************************************************************************

\chapter{Conclusion}\label{chapter:Conclusions}

\ifpdf
    \graphicspath{{Chapter8/Figs/Raster/}{Chapter8/Figs/PDF/}{Chapter8/Figs/}}
\else
    \graphicspath{{Chapter8/Figs/Vector/}{Chapter8/Figs/}}
\fi

In this work we have studied the impact of centre vortices on the Landau gauge gluon propagator by calculating the gluon propagator on original, vortex removed and vortex only lattice configurations. We observe that vortices effectively partition the propagator into long-distance and short-distance strength, with the vortex only configurations encapsulating much of the non-perturbative behaviour of the propagator. This partitioning is consistent with the vortex modified gauge potential representing orthogonal degrees of freedom. Although the vortex only propagator does not demonstrate the full infrared strength of the original propagator, it is clear that vortices are crucial to the long-range behaviour of the propagator.\\

We then investigated the effect of smoothing on the gluon propagator, determining that both three-loop cooling and over-improved stoutlink smearing produce similar amplification of infrared strength and supression of high frequency modes. Indeed, we found that for each sweep of cooling, four sweeps of smearing produces a remarkably similar propagator when calculated on the original configurations. Using cooling as our smoothing method of choice, we found that cooling brings the gluon propagator on the untouched and vortex only configurations closer together. By using the average action as a measure of roughness, we see that it is possible to recover the infrared strength of the vortex-only propagator when compared to the untouched propagator. The accuracy to which it is possible to recreate the gluon propagator form vortex only configurations on similarly smooth original configurations is of particular note.\\

We then presented a novel method for visualising projected vortex configurations, allowing for analysis of vortex geometry in a highly interactive manner. From these visualisations we discover that, at low temperature, vortices tend to span the full lattice extent. Furthermore, we observe an abundance of branching/monopole points present in the vortex structure. By overlaying the topological charge density, we note a distinct relationship between topological charge and singular points where the vortex sheets span all four dimensions. We also find an explicit correlation between vortex locations and regions of high topological charge. Under cooling, we find that the vortex structure of the lattice is considerably simplified, with the residual vortex matter showing a stronger correlation to regions of high topological charge. This indicates that cooling does indeed preserve genuine topological objects relevant to the long distance behaviour of QCD, whilst filtering out extraneous structures. Although many of our findings are somewhat qualitative, these visualisations open many new avenues for investigation of the vacuum structure of QCD and the significance of centre vortices.\\

These findings suggest many potential directions for future work. It would be valuable to investigate whether improved vortex identification allows us to account for the discrepancy between the untouched and vortex only propagators. It is also of interest to calculate the gluon propagator in full QCD, where it is anticipated that we would observe infrared screening due to the presence of fermions. The relevance of smoothing is currently poorly understood, and studying the behaviour of the gluon propagator in the continuum limit would clarify whether smoothing is essential to grow vortices to a physical size, or if it is simply necessary to remove high-frequency fluctuations present on the vortex-only configurations. Developing techniques to identify singular points within our visualisations and quantitatively assessing their relationship with topological charge would also be of great interest.\\

The centre vortex model has demonstrated remarkable success in recent years, and the work presented here continues to support the relevance of centre vortices to the essential properties of confinement and dynamical chiral symmetry breaking in QCD. It is increasingly apparent that centre vortices are an essential component in a complete understanding of QCD vacuum structure.

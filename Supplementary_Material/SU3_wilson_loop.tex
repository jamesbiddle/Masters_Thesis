\documentclass{article}
\usepackage{amsmath}

\begin{document}
Consider a square of dimensions $L\times L$, pierced by $2N$ vortices. Assuming there are an approximately equal number of $+1$ and $-1$ vortices piercing this square, the probability of $n$ $+1$ or $-1$ vortices piercing some area $A \in L^2$ is
\begin{equation}
P_N(n) = {N\choose n} \left(\frac{A}{L^2}\right)^n \left(1-\frac{A}{L^2}\right)^{N-n}\, .
\end{equation}
Therefore, by assuming the different vortex phases are uncorrelated, the expectation value of the Wilson loop can be written as
\begin{equation}
\langle W(\partial A)\rangle = \sum_{n=0}^N \left(\exp\left(\frac{2\pi i}{3}\right)\right)^n\,P_{N}(n)\,\sum_{m=0}^{N}\left(\exp\left(\frac{-2\pi i}{3}\right)\right)^m\,P_{N}(m)\, .
\end{equation}
Taking just the first sum, it evaluates to
\begin{align*}
\left(1-\frac{A}{L^2}\right)^{N}\sum_{n=0}^{N} {N\choose n} \left(\exp\left(\frac{2\pi i}{3}\right)\,\frac{A}{L^2}\left(1-\frac{A}{L^2}\right)^{-1}\right)^n=\left(1+\left(\exp\left(\frac{2\pi i}{3}\right) - 1\right)\frac{A}{L^2}\right)^N\, . 
\end{align*}
So the total expectation value is
\begin{align*}
\langle W(\partial A)\rangle &=\left(1+\left(\exp\left(\frac{2\pi i}{3}\right) - 1\right)\frac{A}{L^2}\right)^N\, \left(1+\left(\exp\left(\frac{-2\pi i}{3}\right) - 1\right)\frac{A}{L^2}\right)^N\\
&=\left(1 -3\frac{A}{L^2} + 3\left(\frac{A}{L^2}\right)^2\right)^N\\
&= \left(\left(\frac{A}{L^2}\right)^3+\left(1-\frac{A}{L^2}\right)^3\right)^N\, .
\end{align*}
Rewriting this in terms of the vortex density $\rho = \frac{N}{L^2}$, we have
\begin{equation}
\langle W(\partial A)\rangle = \left(\left(\frac{A\rho}{N}\right)^3+\left(1-\frac{A\rho}{N}\right)^3\right)^N\, .
\end{equation}
Taking the limit as $N,L^2\rightarrow\infty$, keeping $\rho$ constant, this becomes
\begin{equation}
\langle W(\partial A)\rangle = \exp(-3\rho A)\, .
\end{equation}
Hence we have a simple model for area-law confinement from the vortex model.
\end{document}
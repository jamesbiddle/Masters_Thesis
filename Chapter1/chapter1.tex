%!TEX root = ../thesis.tex
%*******************************************************************************
%*********************************** First Chapter *****************************
%*******************************************************************************

\chapter{Introduction}
\ifpdf
    \graphicspath{{Chapter1/Figs/Raster/}{Chapter1/Figs/PDF/}{Chapter1/Figs/}}
\else
    \graphicspath{{Chapter1/Figs/Vector/}{Chapter1/Figs/}}
\fi
The standard model of particle physics is one of the great accomplishments of modern physics, unifying three of the four forces of nature into one coherent theory. However, despite the remarkable power of the standard model, there are still unsolved questions that exist within its framework. In particular, the theory of the strong interaction, Quantum ChromoDynamics (QCD), is notorious for its mathematical intractability. QCD is the theory governing the behaviour of quark colour interactions, mediated by the exchange of the force-carrying gauge boson known as the gluon. This interaction is responsible for binding quarks into baryons (quark triplets) and mesons (quark/anti-quark pairs). Due to the large coupling constant and non-Abelian nature of the $SU(3)$ gauge group, the techniques of perturbation theory that have proven so successful for performing Quantum ElectroDynamics (QED) calculations cannot be utilised when studying the low-energy behaviour of QCD. Instead, new approaches have been constructed to facilitate an understanding of this fundamental force.\\

First proposed in 1974~\cite{Wilson:1974sk}, the primary technique used to perform QCD calculations is known as the lattice. Rather than treat spacetime as a set of continuous axes, it is instead discretised into a finite number of points on a four-dimensional hypercube. With spacetime reduced to a finite number of points, it becomes possible to perform first-principles calculations, albeit with the introduction of systematic errors that must be accounted for. Utilising the lattice framework and the continual increase in computing power available to researchers, it has become possible over the last 40 years to simulate the behaviour of QCD. These results have proven invaluable in developing an understanding of QCD and in guiding the direction of experiments.\\

Experimental observations have found two key low-energy properties of the strong interaction that must somehow arise from the theory of QCD, namely
%
\begin{enumerate}
\item Confinement of quarks, in which quarks are never observed in isolation.
\item Dynamical chiral symmetry breaking, leading to dynamical mass generation that results in hadrons exhibiting a mass greater than the sum of their quark components.
\end{enumerate}
%
Numerous theories have been proposed to explain how QCD implies the emergence of these properties, and one that has shown particular promise in recent years is the centre vortex model. This model proposes that the vacuum is percolated by topologically non-trivial sheet-like objects known as centre vortices that naturally give rise to confining behaviour. Through lattice calculations it has been possible to investigate the impact of centre vortices on QCD. The results of these calculations have been very promising, suggesting an intimate relationship between centre vortices and the properties of confinement and dynamical chiral symmetry breaking. Continuing this line of investigation, part of this research is devoted to exploring the impact of centre vortices on the gluon propagator. The gauge boson propagator is an essential building block of any gauge theory, and an understanding of its behaviour is key to a full understanding of the theory. The gluon propagator can also be compared to the well understood photon propagator, allowing for a direct comparison to the non-confining theory of QED.\\

We also present a new approach for studying the centre vortex model. By making use of centre vortex identification techniques and 3D visualisation software, it becomes possible to construct 3D models of centre vortices on the lattice. These models enable us to explore the properties of centre vortices in a hands-on manner, allowing for an intuitive graphical understanding of this theoretical model. They also reveal the geometrical properties of centre vortices, raising interesting questions about the link between vortex geometry and the properties of QCD.\\

This thesis is structured as follows: Chapter~\ref{chapter:LatticeQCD} will review QCD in the continuum and demonstrate how this theory can be reformulated on the lattice. Chapter~\ref{chapter:Topology} will describe in detail the centre vortex model, vortex identification and a brief discussion of other topological objects. Chapter~\ref{chapter:GluonPropagator} will detail the calculation of the gluon propagator and the data analysis techniques used in this work. Chapter~\ref{chapter:Smoothing} provides an introduction to the smoothing routines used to study topological objects on the lattice. Chapter~\ref{chapter:GluonPropagatorResults} contains the results from the gluon propagator on vortex modified backgrounds and Chapter~\ref{chapter:Visualisations} presents the visualisations of the centre vortex vacuum. Finally, Chapter~\ref{chapter:Conclusions} summarises the findings of this research.
%!TEX root = ../thesis.tex
%*******************************************************************************
%****************************** Third Chapter **********************************
%*******************************************************************************
\chapter{Topology of the Lattice}

% **************************** Define Graphics Path **************************
\ifpdf
    \graphicspath{{Chapter3/Figs/Raster/}{Chapter3/Figs/PDF/}{Chapter3/Figs/}}
\else
    \graphicspath{{Chapter3/Figs/Vector/}{Chapter3/Figs/}}
\fi
\section{Confinement}
The confinement property of QCD, and the accompanying notion of asymptotic freedom, is one of the defining low-energy features of the theory of the strong interaction. With Gell-mann and Zweig's concurrent proposal of quarks as the elementary constituents of baryons and mesons~\cite{GellMann:1964nj,Zweig:1964jf}, it is natural to then attempt to observe these new particles in isolation. However, efforts to observe quarks proved impossible. Early experiments testing the behaviour of electron-proton collisions demonstrated that protons scatter elastically, behaving as though they are finite-sized particles recoiling electromagnetically from the incident electron~\cite{Hofstadter:1956qs}. These experiments indicated no further substructure to the proton, inconsistent with the quark model. As accelerator energies improved, later experiments~\cite{Bloom:1969kc, Breidenbach:1969kd}, using electron energies of $7$ and $10~\si{GeV}$ found that inelastic scattering effects became dominant, with electrons behaving as though they were scattering off of loosely bound constituent particles. To explain this behaviour, Feynman proposed what is known as the 'parton' model~\cite{Feynman:1969ej}, treating the proton as being comprised of non-interacting electrically charged particles in the limit that the incident electron energy tends towards infinity. This is precisely the notion of asymptotic freedom; at large distance scales the partons are tightly bound, whereas at short distances they behave as free particles. It did not take long for the separate theories of quarks and partons to recognised as complementary, and by the early 70's the quark-parton model of hadrons accurately explained the the experimental results observed in particle colliders.\\

These experimental and theoretical results led in part to the development of the non-Abelian gauge field theory of QCD, as introduced in Chapter~\ref{chapter:LatticeQCD}. Evidence that non-Abelian gauge theories are asymptotically free was demonstrated in 1973~\cite{Gross:1973id}, and experimental evidence of the existence of 3 quark colours through study of the cross section of $e^+ e^-$ collisions reinforces the initial $SU(3)$ colour symmetry anticipated by Gell-Mann and Zweig. At high energies, QCD has consistently explained the behaviour of hadronic matter, and became the accepted theory of the strong nuclear interaction. However, the question of whether QCD is indeed a confining theory still remains. As confinement is a low-momentum property of QCD, it is apparent that any analytic proof of confinement must take place far from the asymptotic limit. To date, no such first-principles proof has been found. 
\section{Centre Vortices}
Originally proposed by 't Hooft in 1978, centre vortices 
\subsection{Maximal Centre Gauge}\label{sec:MCG}
\subsection{Centre Projection}
\section{Instantons}
The notion of a non-empty QCD vacuum is intrinsically tied to the existence of instanton solutions to the gluonic equations of motion. An instanton solution is simply a non-trivial solution to the equations of motion that has zero total energy
\subsection{Topological Charge}

%A frequently seen mistake is to use `\textbackslash begin\{center\}' \dots `\textbackslash end\{center\}' inside a figure or table environment. This center environment can cause additional vertical space. If you want to avoid that just use `\textbackslash centering'
%
%
%\begin{table}
%\caption{A badly formatted table}
%\centering
%\label{table:bad_table}
%\begin{tabular}{|l|c|c|c|c|}
%\hline 
%& \multicolumn{2}{c}{Species I} & \multicolumn{2}{c|}{Species II} \\ 
%\hline
%Dental measurement  & mean & SD  & mean & SD  \\ \hline 
%\hline
%I1MD & 6.23 & 0.91 & 5.2  & 0.7  \\
%\hline 
%I1LL & 7.48 & 0.56 & 8.7  & 0.71 \\
%\hline 
%I2MD & 3.99 & 0.63 & 4.22 & 0.54 \\
%\hline 
%I2LL & 6.81 & 0.02 & 6.66 & 0.01 \\
%\hline 
%CMD & 13.47 & 0.09 & 10.55 & 0.05 \\
%\hline 
%CBL & 11.88 & 0.05 & 13.11 & 0.04\\ 
%\hline 
%\end{tabular}
%\end{table}
%
%\begin{table}
%\caption{A nice looking table}
%\centering
%\label{table:nice_table}
%\begin{tabular}{l c c c c}
%\hline 
%\multirow{2}{*}{Dental measurement} & \multicolumn{2}{c}{Species I} & \multicolumn{2}{c}{Species II} \\ 
%\cline{2-5}
%  & mean & SD  & mean & SD  \\ 
%\hline
%I1MD & 6.23 & 0.91 & 5.2  & 0.7  \\
%
%I1LL & 7.48 & 0.56 & 8.7  & 0.71 \\
%
%I2MD & 3.99 & 0.63 & 4.22 & 0.54 \\
%
%I2LL & 6.81 & 0.02 & 6.66 & 0.01 \\
%
%CMD & 13.47 & 0.09 & 10.55 & 0.05 \\
%
%CBL & 11.88 & 0.05 & 13.11 & 0.04\\ 
%\hline 
%\end{tabular}
%\end{table}
%
%
%\begin{table}
%\caption{Even better looking table using booktabs}
%\centering
%\label{table:good_table}
%\begin{tabular}{l c c c c}
%\toprule
%\multirow{2}{*}{Dental measurement} & \multicolumn{2}{c}{Species I} & \multicolumn{2}{c}{Species II} \\ 
%\cmidrule{2-5}
%  & mean & SD  & mean & SD  \\ 
%\midrule
%I1MD & 6.23 & 0.91 & 5.2  & 0.7  \\
%
%I1LL & 7.48 & 0.56 & 8.7  & 0.71 \\
%
%I2MD & 3.99 & 0.63 & 4.22 & 0.54 \\
%
%I2LL & 6.81 & 0.02 & 6.66 & 0.01 \\
%
%CMD & 13.47 & 0.09 & 10.55 & 0.05 \\
%
%CBL & 11.88 & 0.05 & 13.11 & 0.04\\ 
%\bottomrule
%\end{tabular}
%\end{table}
